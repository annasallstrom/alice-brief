\documentclass{scotus}
\usepackage[ignoremissing]{bluebook}
\usepackage{ifxetex}
\usepackage{xspace}

\usepackage{listings}
%
% This is a really awful hack that causes the listings package to output double
% quotation marks for strings with proper directional quotation marks.
%
\makeatletter
\def\lst@Append#1{%
    \advance\lst@length\@ne
    \if\noexpand#1"%
        \lst@hack@quote
    \else
        \lst@token=\expandafter{\the\lst@token#1}%
    \fi
}
\newif\if@lst@hack@inquote
\def\lst@hack@quote{%
    \if@lst@hack@inquote
        \global\@lst@hack@inquotefalse
        \lst@token=\expandafter{\the\lst@token''}%
    \else
        \global\@lst@hack@inquotetrue
        \lst@token=\expandafter{\the\lst@token``}%
    \fi
}
\makeatother


\defcase{i4i}{
p=Microsoft Corp.,
d=i4i Limited Partnership,
cite=131/S. Ct./2238,
year=2011
}

\defcase{bilski}{
p=Bilski,
d=Kappos,
cite=130 S. Ct. 3218,
year=2010
}

\defstatcode{112p2}{
cite={35/U.S.C./S 112, \P~2},
year=2011
}

\defcase{quanta}{
p={Quanta Computer, Inc.},
d=LG Electronics Corp.,
cite=553/U.S./617,
year=2008
}

\defstatcode{101}{
cite=35 U.S.C. S 101,
year=2013
}

\defcase{flook}{
p=Parker,
d=Flook,
cite=437 U.S. 584,
year=1978,
inlinedefendant
}

\defcase{funkbros}{
p=Funk Brothers Seed Co.,
d=Kalo Inoculant Co.,
cite=333 U.S. 127,
year=1948
}

\defcase{leroy}{
p=Le Roy,
d=Tatham,
year=1852,
cite=55 U.S. (14 How.) 156
}

\defcase{benson}{
p=Gottschalk,
d=Benson,
inlinedefendant,
cite=409 U.S. 63,
year=1972
}

\defcase{mayo}{
p=Mayo Collaborative Services,
d={Prometheus Laboratories, Inc.},
cite=132 S. Ct. 1289,
inline=Mayo,
year=2012
}

\defcase{cls-enbanc}{
p=CLS Bank International,
d=Alice Corp.,
cite=717 F.3d 1269,
year=2013,
court=Fed. Cir.,
enbanc
}

\defpatent{pat375}{
number={7725375},
date={June 27, 2005}
}

\ExplanatoryPhrase{vacated and rehearing en banc granted,}

\defcase{cls-panel}{
p=CLS Bank International,
d=Alice Corp.,
cite=685 F.3d 1341,
year=2012,
court=Fed. Cir.,
paren=panel decision,
subsequent={vacated and rehearing en banc granted, cls-vacate}
}
\defcase{cls-vacate}{
sameparties=cls-panel,
cite=484 Fed. Appx. 559,
year=2012,
court=Fed. Cir.
}

\defcase{diehr}{
p=Diamond,
d=Diehr,
cite=450 U.S. 175,
inlinedefendant,
year=1981
}
\defbook{mpep}{
    title=Manual of Patent Examining Procedure,
    instauth=United States Patent and Trademark Office,
    year=2012,
    edition={8th ed., 9th rev.},
    hereinafter=MPEP
}

\defcasedoc{cls-dist95}{
sameparties=cls-panel,
docket=1:07-cv-974,
cite=768 F. Supp. 2d 221,
docname=Alice Corp. Pty. Ltd.'s Renewed Cross-Motion for Partial Summary
Judgment as to Subject Matter Eligibility,
date={Sept. 22, 2010},
court=Dist. D.C.,
paren=Doc.\ No.\ 95
}
\defcasedoc{cls-dist53}{
sameparties=cls-panel,
docket=1:07-cv-974,
cite=768 F. Supp. 2d 221,
court=Dist. D.C.,
docname=Alice Corp. Pty. Ltd.'s Cross-Motion for Partial Summary
Judgment as to Subject Matter Eligibility,
date={Apr. 3, 2009},
paren=Doc.\ No.\ 53
}
\defcasedoc{cls-dist54}{
sameparties=cls-panel,
docket=1:07-cv-974,
cite=768 F. Supp. 2d 221,
docname=Alice Corp. Pty. Ltd.'s Cross-Motion for Partial Summary
Judgment as to Subject Matter Eligibility,
date={Apr. 3, 2009},
court=Dist. D.C.,
paren=Doc.\ No.\ 54
}
\defcasedoc{cls-dist68}{
sameparties=cls-panel,
docket=1:07-cv-974,
cite=768 F. Supp. 2d 221,
court=Dist. D.C.,
docname=Alice Corp. Pty. Ltd.'s Reply Memorandum in Support of its Cross-Motion
for Partial Summary Judgment as to Patent-Eligibility,
date={May 19, 2009},
paren=Doc.\ No.\ 68
}
\defcasedoc{cls-dist99}{
sameparties=cls-panel,
docket=1:07-cv-974,
cite=768 F. Supp. 2d 221,
court=Dist. D.C.,
docname=Alice Corp. Pty. Ltd.'s Reply Memorandum in Support of its Renewed
Cross-Motion for Partial Summary Judgment as to Patent-Eligibility,
date={Oct. 22, 2010},
paren=Doc.\ No.\ 99
}
\defcasedoc{cls-app22}{%
sameparties=cls-panel,
docket=11-1301,
cite=685 F.3d 1341,
court=Fed. Cir.,
docname=Brief of Defendant-Appellant Alice Corp. Pty. Ltd.,
date={June 24, 2011},
paren=Doc.\ No.\ 22
}
\defcasedoc{cls-app33}{%
sameparties=cls-panel,
docket=11-1301,
cite=685 F.3d 1341,
court=Fed. Cir.,
docname=Reply Brief of Defendant-Appellant Alice Corp. Pty. Ltd.,
date={Sept. 16, 2011},
paren=Doc.\ No.\ 33
}
\defcasedoc{cls-app194}{
sameparties=cls-panel,
docket=11-1301,
cite=717 F.3d 1269,
court=Fed. Cir.,
docname=\emph{En Banc} Response Brief of Defendant-Appellant Alice Corp. Pty.
Ltd.,
date={Jan. 10, 2013},
paren=Doc.\ No.\ 194
}

\newcounter{claimcounter}
\def\theclaim{%
A data processing system to enable the exchange of an obligation
between parties, the system comprising:
  \element a communications controller,
  \element a first party device, coupled to said communications controller,
  \element a data storage unit having stored therein
    \element\sub (a)\ind information about a first account for a first party,
    independent from a second account maintained by a first exchange
    institution, and
    \element\sub (b)\ind information about a third account for a second party,
    independent from a fourth account maintained by a second exchange
    institution; and
  \element a computer, coupled to said data storage unit and said
  communications controller, that is configured to
    \element\sub (a)\ind receive a transaction from said first party device via
    said communications controller;
    \element\sub (b)\ind electronically adjust said first account and said
    third account in order to effect an exchange obligation arising from said
    transaction between said first party and said second party after ensuring
    that said first party and/or said second party have adequate value in said
    first account and/or said third account, respectively; and
    \element\sub (c)\ind generate an instruction to said first exchange
    institution and/or said second exchange institution to adjust said second
    account and/or said fourth account in accordance with the adjustment of said
    first account and/or said third account, wherein said instruction being an
    irrevocable, time invariant obligation placed on said first exchange
    institution and/or said second exchange institution.
}
\newcount\claimeltcount
\def\claimelt#1{%
    \begingroup
        \let\element\or
        \let\sub\relax
        \let\preamble\relax
        \let\ind\space
        \claimeltcount=#1\relax
        \expandafter\ifcase\expandafter\claimeltcount\theclaim\fi
        \unskip
    \endgroup
}
\def\wholeclaim{%
    \begingroup
        \let\preamble\par
        \def\element{\par\endgroup\begingroup\advance\leftskip\parindent}%
        \def\sub{\advance\leftskip\parindent}%
        \let\ind\quad
        \begingroup\theclaim\par\endgroup
    \endgroup
}


% This sets up various macros used in this brief.
\setcounter{passimnum}{5}

%
% If you do not have these fonts, comment the next few lines out by changing
% \ifxetex to \iffalse
%
\ifxetex
    \usepackage{fontspec}
    \setmainfont[BoldFont=ITC Century Std-Bold,Ligatures=TeX]{%
        Century Expanded LT Std%
    }
    \def\titlefont{\fontspec{Cloister Black}\LARGE}
    \def\basickeywordstyle{\addfontfeatures{Letters=UppercaseSmallCaps}}
\else
    \def\basickeywordstyle{\bfseries}
\fi

%
% Macros for how long the computer program is
%
\def\numlines{seven\xspace}
\def\Numlines{Seven\xspace}
\def\wholeprogram{\programlines{1}{7}}

%
% Macros to format the claim language and/or computer program
%
\def\programlines#1#2{%
    \lstset{keywordstyle=\basickeywordstyle,identifierstyle=\itshape}
    \lstset{breaklines=true,showstringspaces=false}
    \lstset{aboveskip=0pt,belowskip=0pt}
    \lstset{language=[Visual]Basic,columns=fullflexible}
    \lstinputlisting[firstline=#1,lastline=#2]{%
        implementations/375claim26.bas%
    }%
}
\long\def\inlinebox#1{%
    \vskip\parskip
    \hrule\kern1.5pt \hrule\nobreak
    \begingroup
        \vskip 4pt
        \parskip=0pt
        #1\par
        \nobreak\vskip 4pt
    \endgroup
    \hrule\kern1.5pt \hrule
}
\long\def\claimtext#1#2{%
    \noindent
    \bblabel{appx:claim:#1}%
    \textsc{Claim 26, #1:}\par
    {\itshape#2\unskip}%
}
\long\def\claimbox#1#2{%
    \vskip.5\baselineskip
    \inlinebox{%
        \claimtext{#1}{#2}%
    }%
}
\long\def\codeandclaimbox#1#2#3#4{%
    \vskip.5\baselineskip
    \inlinebox{%
        \claimtext{#1}{#2}%
        \vskip 4pt
        \hrule
        \vskip 4pt
        \noindent
        \ifnum#3=#4\relax
            \textsc{Computer code, line #30:}\par
        \else
            \textsc{Computer code, lines #30--#40:}\par
        \fi
        \programlines{#3}{#4}%
    }%
}%

%
% Other useful macros
%
\def\Amici{\emph{Amici}\xspace}
\def\amici{\emph{amici}\xspace}

\endinput


\docket{13-298}
\casecaption{
\textsc{Alice Corporation Pty. Ltd.},\\\The{Petitioner,}\\v.\\
\textsc{CLS Bank International and CLS Services Ltd.},\\
\The{Respondent.}
}

\posture{On Writ of Certiorari \\
to the United States Court of Appeals \\
for the Federal Circuit}

\counsel{
\textsc{Charles Duan} \\
\counselofrecord \\
\textsc{Public Knowledge} \\
1818 N St NW, Suite 410 \\
Washington, DC 20036 \\
(202) 861-0020 \\
cduan@publicknowledge.org \\
\and
\textsc{Jack Lerner} \\
\textsc{USC Intellectual Property and Technology Law Clinic} \\
699 Exposition Blvd, Room 425 \\
Los Angeles, CA 90089 \\
(213) 740-2537 \\
iptlcsct@law.usc.edu
}
\counselis{Counsel for Amici Curiae}

\title{Brief of Public Knowledge and the Application Developers Alliance as
\protect\\\emph{Amici Curiae} in Support of Respondent}

\begin{document}

\maketitle

\romanpagenumbers
\tableofcontents

\part{Table of Authorities}

\tableofauthorities

\clearpage
\arabicpagenumbers

\part{Interest of \emph{Amici Curiae}}

Public Knowledge is a non-profit organization that is dedicated to preserving
the openness of the Internet and the public's access to knowledge; promoting
creativity through balanced intellectual property rights; and upholding and
protecting the rights of consumers to use innovative technology lawfully. As
part of this mission, Public Knowledge advocates on behalf of the public
interest for a balanced patent system, particularly with respect to new and
emerging technologies.\footnote{Per Supreme Court Rule 37(6), no counsel for a
party authored this brief in whole or in part, and no counsel or party made a
monetary contribution intended to fund the preparation or submission of the
brief. No person or entity, other than \amici, their members, or their counsel,
made a monetary contribution to the preparation or submission of this brief. Per
Rule 37(3)(a), consent has been granted for the filing of this brief, as
indicated by the blanket consents from counsel for petitioner and counsel for
respondents docketed December 11, 2013.}

Public Knowledge has previously served as \emph{amicus} in key patent cases.
\sentence{e.g., i4i; bilski; quanta}.

The Application Developers Alliance (ADA) is a nonprofit industry association
comprising more than 25,000 individual software developers and more than 135
companies who design and build applications (“apps”) for consumers to use on
mobile devices like smartphones and tablets. Apps run on software platforms,
including Google’s Android, Apple’s iOS, and Facebook, and are sold or
distributed through virtual stores like Google’s Play Store. ADA is dedicated to
meeting the needs of app developers as creators, innovators, and entrepreneurs,
by delivering essential information and resources and by advocating for public
policies that promote the app ecosystem.

App developers are both central to innovation and vulnerable to the patent laws
that surround innovation. By innovating rapidly and cheaply, app developers
represent an increasingly robust force in the economy. The app economy is now
globally valued at over \$53 billion and has
created approximately 466,000 jobs in the United States since
2007.\footnote{\sentence{pappas; mandel at 13}.} But many app developers,
including
ADA members, are struggling as a result of abusive patent assertion, especially
that originating from patent assertion entities (PAEs). Such entities often
assert overly broad patents, propounding unfounded infringement allegations and
aggressive litigation threats, which deeply chill innovation.

Inconsistency and uncertainty in areas of patent law, such as subject matter
eligibility, give rise to many of the patents asserted by PAEs, which are
directed to ineligible ideas cloaked in eligible-sounding language. This forces
app developers to conclude that innovation is not worth the expensive baggage of
defending against such claims, resulting in delays to and deficiencies in app
development and overall innovation.\footnote{\sentence{see, e.g., chien-oti at
17}.} Thus,
ADA and its members have a strong interest in this Court providing clarity in
this area of patent law.



\fullcite{i4i}
\fullcite{bilski}
\fullcite{quanta}

\clearpage

\part{Summary of Argument}

Abstract ideas are not eligible for patenting because, as this Court has
steadfastly maintained, certain fundamental subject matter must be fixed in the
public domain, so that patents may serve their constitutional mandate to
``promote the Progress of Science and the useful Arts.'' Being the basic tools
of innovation, abstract ideas must remain available to the public; to do
otherwise would impede innovation more than promote it.

The case today tests how far a patent may encroach on that valuable domain
reserved to innovators, creators, and the public. Petitioner holds patents
to computer technology. The patent claims at issue are
lengthy and detailed, some with over two hundred words. But those claims
actually cover very simple ideas. The verbose language is a mere facade
masking basic concepts.

To demonstrate this, \amici have implemented one of those 200-word
claims---in only 7 lines of computer code.

This computer program implementation shows that the patent
claims are directed to nothing more than an abstract idea implemented on a
general-purpose computer, which should not be patent-eligible.
%
% Summary of I.C argument
To hold otherwise would contravene
Court's precedent and undermine the rationale for unpatentability of
abstract ideas. Such ``abstract-idea-plus-computer'' patents would be effective
monopolies on the basic tools of innovation, a result that the
Court has adamantly rejected.

To prevent further errors of this sort, \amici identify points of clarification
on the law of subject matter eligibility. Enunciating these specific points not
only will correct the judgment below and guide the lower courts, but also will
ensure that those valuable basic tools of innovation remain available to all.

\part{Argument}

This case presents the recurrent question of what constitutes patentable
subject matter, particularly with regard to the fields of computer software and
business methods. \Amici address two aspects of this question as they relate to
the present case. First, this brief shows that the patented claims at issue are
directed to ineligible abstract ideas, by
implementing one of those claims is used to assist in this demonstration.
Second, in view of the fractured opinions of the Federal Circuit below, \amici
suggests principles for guiding the lower courts in deciding future cases.



%%%%%%%%%%%%%%%%%%%%%%%%%%%%%%%%%%%%%%%%%%%%%%%%%%%%%%%%%%%%%%%%%%%%%%%%
%
% SECTION I
%
\section{The Claims at Issue Are Ineligible for Patenting Because They
Preempt an Abstract Idea}

The question presented is whether Petitioner's claims are directed to
patent-eligible subject matter. Generally, ``any new and
useful process, machine, manufacture, or composition of matter'' is eligible for
patenting. \sentence{101}. But three exceptional fields
are nevertheless ineligible: laws of nature, physical phenomena, and abstract
ideas. \sentence{e.g., mayo at 1293 (quoting \sentence{diehr at 185})}.

The claims of these patents, as with many patents in the computer technology
field, are full of complex technical language.
But these claims actually present vary basic concepts---so basic, in fact, that
\amici have prepared a \numlines-line computer program
that implements all the features of one of the most complex claims. The program
demonstrates that the claims recite not specialized, technical systems, but
rather a broad, general, simple algorithm that reduces to nothing more than an
abstract idea run on a computer. Because mere application of a general-purpose
computer should not render an otherwise abstract idea patentable, \amici
urge the Court to find the present claims ineligible.



%%%%%%%%%%%%%%%%%%%%%%%%%%%%%%%%%%%%%%%%%%%%%%%%%%%%%%%%%%%%%%%%%%%%%%%%
%
% SECTION I.A
%
\subsection{The Claims Can Be Implemented in Just \Numlines Lines of
Computer Code}

Much of the disagreement in the lower court's fractured decision stemmed
from a disagreement over the nature of the patent claims at issue. Judge Lourie,
writing for five judges,
found the claims to recite ``a handful of
computer components in generic, functional terms that would encompass any
device'' and unduly preempt an abstract idea. \sentence{see cls-enbanc at 1290}.
Judge Rader, writing for four judges, found those same claims narrowly tailored,
``limited to an implementation that includes at least four separate structural
components'' rendering the claim patent-eligible. \sentence{see cls-enbanc at
1307}.

Here, the claims at issue use complex, technical-sounding language, making them
appear to be directed to a narrowly tailored invention. One of the claims at
issue recites, among other things, a ``communications controller,'' a
``data storage unit,'' and an ``instruction being an irrevocable, time invariant
obligation.'' \sentence{pat375 at claim 26, col. 66-67}.\footnote{Claim 26 of
\inline{pat375} is considered in this brief because it was found patentable by
the greatest number of judges of the lower court decision. \sentence{see
cls-enbanc at 1309 (Rader, Linn, Moore \& O'Malley, JJ.);
cls-enbanc at 1327 (Newman, J.)}.
\Amici could have easily used any other claim at issue.
For reference, Claim
26 is reprinted in \inline{this at appx:claim}.}

But beneath this veneer
of technical language is a very simple, basic idea being patented. As a
demonstration, the computer program shown in Figure~\ref{code-listing}
implements all the features of Claim 26 of \inline{pat375}.

\def\floatpagefraction{.1}
\begin{figure}[p]
\inlinebox{\wholeprogram}%
\vskip2\baselineskip
\caption{Implementation of Claim 26 of \protect\inline{pat375}.}
\label{code-listing}
\end{figure}

\Amici selected the \textsc{basic} programming language for this program both
because it is simple to understand and because it predates the
patent.\footnote{The earliest
possible priority date of the patent is 1992. The \textsc{basic} language dates
back to 1964. \sentence{see basic-manual}. Thus,
the computer techniques used in this brief were
``well-understood, routine, conventional activity
previously engaged in by researchers in the field'' as of the priority date of
the patent. \sentence{mayo at 1294; cf.
cls-enbanc at 1310 (Rader, J.) (asserting that the use of computers in the
claims did not involve such conventional activity)}.}
A complete explanation of the working of
this program as it relates to Claim 26 of \inline{pat375} is presented in
\inline{this at appx:code}.

As the Court will observe, the computer program is only \numlines lines long,
indicating that
the verbose language of the claims does not in fact demand specific, particular
implementations but rather expansively preempts all uses of a simple, basic
idea.

Certain
\bblabel{computer-quotes}
judges below were misled by the language of the
claims and the patent.
Judge Rader believed that Claim 26 ``covers the
use of a computer and
other hardware specifically programmed to solve a complex problem'' through the
use of ``at least four separate structural components.'' \sentence{cls-enbanc at
1307}. He reviewed the ``at least thirty two figures
which
provide detailed algorithms'' to conclude that ``[l]abeling
this system claim an `abstract concept' wrenches all meaning from those words.''
\sentence{cls-enbanc at 1309}. Judge Moore similarly found a
similarly-worded claim ``limited to one that is configured to perform certain
functions in a particular fashion'' and, based on one of the
flowcharts, suggested that the claims demanded a dizzyingly long
and complex algorithm. \sentence{cls-enbanc at 1318}. And Judge Linn
concluded that, while they may be based
on an abstract idea, ``the claims here are directed to very specific ways of
doing that.''
\sentence{cls-enbanc at 1741}.

The common thread among all of these judges is an assumption that, given the
heavy use of technical language in the specification and claims, only a
specific, complex, technical computer program could infringe the patents. As the
above \numlines-line computer program demonstrates, this assumption was in
error.

The computer program devised by \amici reads
the claim as a whole, as this Court requires. \sentence{see diehr at 188}. As
the detailed appendix shows, every
claim limitation is considered and implemented appropriately in the
computer code, so it cannot be said that details or limitations have been
stripped from the claim. \sentence{see this at appx:code; cf. diehr at 188 (``It
is inappropriate to dissect the claims into old and new elements and then to
ignore the presence of the old elements in the analysis.'')}. Furthermore,
because
the computer program is a functional, working implementation of the claim, it
cannot be argued that it is a mere abstraction or generalization of the claims.

Thus, Claim 26 of \inline{pat375} is directed not to a complex
system requiring specialized hardware, but rather to a basic,
\numlines-line computer algorithm.


%%%%%%%%%%%%%%%%%%%%%%%%%%%%%%%%%%%%%%%%%%%%%%%%%%%%%%%%%%%%%%%%%%%%%%%%
%
% SECTION I.B
%
\subsection{The Claims Cover All
Computer Implementations of an Abstract Idea}
\bblabel[Section \#]{sec1b}

The example computer program shows that the asserted claims, though lengthy and
technical in appearance, are actually directed only to a very simple, basic
computer procedure. \Amici now proceed to use this example computer program to
show that the asserted claims are directed only to the abstract idea of
accounting by a third-party escrow.

This Court's precedent lays out several guidelines for determining whether a
claim is directed to an abstract idea.
``A principle, in the abstract, is a fundamental truth; an original cause; a
motive; these cannot be patented.''
\sentence{benson at 67 (quoting \sentence{leroy at 175})}.
%Claims directed merely to an abstract idea are
%not eligible under \inline{101}, because they are the ``basic tools of
%scientific and technological work,'' \clause{benson at 67}, which therefore
%are ``part of the storehouse of knowledge of all men\ldots free to all men and
%reserved exclusively to none,'' \clause{bilski at 3225 (quoting
%\sentence{funkbros
%at 130}) (omission in original)}. Furthermore, inclusion
Furthermore, ``conventional or
obvious''
post-solution or pre-solution activity cannot
render a claim eligible, because otherwise
``a competent draftsman could attach some form of post-solution activity'' to
``transform an unpatentable principle into a patentable process.''
\sentence{flook at 590; see also mayo at 1300 (holding ``conventional steps,
specified at a high level of generality,'' to similarly not confer patent
eligibility)}.\looseness=-2

Following these guidelines, \amici will analyze Claim 26, line by
line, to determine that every claim element is (1) an inherent aspect of the
abstract idea of third-party escrow, (2) a conventional component of a
general-purpose computer, or
(3) insignificant pre- or post-solution activity.
Thus, following this Court's precedent, \amici will have shown the claim to be
ineligible.


Elements\footnote{Elements will be referenced by numbers corresponding to
the claim reprinted in the appendix. \sentence{see this at appx:claim}.} 1--2 of
the
claim describe ordinary components of a general-purpose computer. \sentence{see
this at appx:claim:elements 1--2}. The ``communications controller'' and
``first party device'' are broad, general terms that encompass basic computer
components for interacting with users.\footnote{\sentence{see turing at 231-232
(describing the Turing machine, a fundamental model for all computers, as
including a ``paper tape'' for communicating with the user)}.} Furthermore,
these two components are only recited in conjunction with a step of receiving
data, which as explained below is insignificant pre-solution activity.
\nofullcite{turing}

Elements 3--5 describe basic record-keeping operations inherent in the idea of
third-party escrow. Although the claim language verbosely describes a ``data
storage unit'' with ``information about a first account'' and second account,
the computer program demonstrates that these elements in fact
require nothing more than recording two
numbers in a computer. \sentence{see this at appx:claim:elements 3--5}.
Certainly one would necessarily store such account information as part of an
escrow service.\footnote{The ``data storage unit'' is an essential part of a
general-purpose computer. \sentence{see turing at 231-232 (further explaining
that the Turing machine includes an \emph{m}-configuration for storing the state
of the machine)}.}

Element 6 recites ``a computer,'' and as such only further describes a
general-purpose computer.

Element 7 states that the computer must ``receive a transaction.'' This Court
and others have repeatedly held that Steps of
obtaining data to be used for processing
constitute insignificant pre-solution activity. \sentence{see, e.g., mayo at
1297-1298 (treating as pre-solution activity a step of determining a level of
metabolites prior to adjusting a treatment); meyer at 794 (``[A] data gathering
step\ldots cannot make an otherwise nonstatutory claim statutory.'')}. As such,
this
claim element does not contribute to the eligibility of the claim.

Element 8 describes two steps to be performed by the computer, both of which are
inherent in the idea of third-party escrow. First, the computer is tasked with
``ensuring that said first party and/or said second party have adequate value''
in their accounts. The computer code shows that this amounts to nothing more
than a comparison, checking whether an account balance is greater than an amount
to be transferred out of that account. \sentence{see this at appx:claim:element
8}. This is the basic purpose of a third-party escrow broker, who
must ensure that the parties' accounts contain sufficient funds.

Second, element 8 requires the computer to ``electronically adjust said first
account and said third account.'' Two lines of computer
code perform this whole operation.
\sentence{see this at appx:claim:element 8}. Again, this
is inherent in any third-party escrow service, which must adjust account balance
records to account for a transaction.

Element 9 instructs that the computer ``generate an instruction to said first
exchange institution and/or said second exchange institution to adjust said
second account and/or said fourth account.'' Despite the fifty-nine-word length
of this element, it reduces to a single operation: printing out a
message describing the transaction that was just completed. \sentence{see this
at appx:claim:element 9}. This elementary output step is quintessential
post-solution activity
that should not contribute to the eligibility of the claim. \sentence{cf.
flook at 590 (treating as post-solution activity a step of adjusting an alarm
limit in response to a computation)}.

This claim is directed to nothing more than an abstract idea of
third-party escrow, in conjunction with insignificant pre-solution and
post-solution activity, and ordinary---albeit verbosely described---components
of
a general purpose computer.
This Court should accordingly hold the claim, and others like it, unpatentable.



%%%%%%%%%%%%%%%%%%%%%%%%%%%%%%%%%%%%%%%%%%%%%%%%%%%%%%%%%%%%%%%%%%%%%%%%
%
% SECTION I.C
%
\subsection{Claims that Preempt Substantially All Computer Implementations of an
Abstract Idea Should Be Ineligible}
\bblabel[Section \#]{sec1c}

% Trying to shorten things
\iffalse
A patent claim with the ``practical effect'' of removing an abstract idea from
the public domain is ineligible under \inline{101}. \sentence{see benson at
71-72}.
Claim 26 of the '375 patent would have the practical effect of removing all uses
of an abstract idea \emph{implemented on a general-purpose computer} from the
public domain.
\fi

In the claims at issue, the Court should hold that the recitation of a
general-purpose computer does not render the claims eligible under \inline{101}.
This follows, first, from the goal of promoting innovation that lies at the
heart of
the Court's \inline{101} doctrine, and second, from the rules of law
the Court has derived from these principles.

The abstract ideas exception is grounded in the principle that certain
fundamental subject matter must be fixed in the public domain, so that
patents may serve their constitutional mandate to ``promote the Progress
of\ldots the useful Arts.'' \sentence{patentclause}.
Abstract ideas are unpatentable because they
are ``the basic tools of scientific and technological work,'' \clause{benson at
67}, and must remain ``free to all men and reserved exclusively to none,''
\clause{bilski at 3218 (quoting \sentence{funkbros at 130})}. ``[M]onopolization
of those tools through the grant of a patent might tend to impede innovation
more than it would tend to promote it.'' \sentence{mayo at 1293; accord bilski
at 3228 (patent law must avoid ``granting
monopolies over procedures that others would discover by independent, creative
application of general principles'')}.

Permitting patents on abstract ideas merely tied to general-purpose computers
would eviscerate this principle.
Computers are in
widespread use today, and they are essential to innovation and a productive
economy. \sentence{see, e.g., brynjolfsson at 4}. 
Allowing patents on abstract ideas merely tied to computers
would relegate
innovators to practicing abstract ideas on pencil and
paper. Needless to say, given the general importance of computers, such an
absurd state of affairs would severely hamstring innovation.
The basic tools of innovation must remain basic tools, available to all even
when they are, or must be, implemented on general-purpose technologies.

As an analogy, consider a patent claim directed to long division performed with
pencil and paper. In theory, long division could be practiced in the mind, but
as a practical matter no ordinary person can do so. Thus, this pencil-and-paper
patent would effectively make the abstract idea of long division unusable.
Similarly, computers are capable of tasks that ordinary
humans cannot perform unaided, even though those tasks may be
abstract ideas. The public must be able to apply these abstract ideas to
computers if those abstract ideas are to
remain ``free to all men and reserved exclusively to none.''\footnote{%
Advances in computer hardware are of course themselves eligible for patent
protection.  \sentence{see cls-enbanc at 1292 (Lourie, J., concurring)
(observing in dicta that computers \emph{per se} are ``surely patent-eligible
machines'')}. Equally so would be advances in pencil technology. But these are
distinguishable from mere annexation of abstract ideas to computers or pencils.}

In view of this important principle, the Court has eschewed
formalistic exegesis in favor of a practical analysis of the actual, effective
scope of the claims.
So, for example,
``limiting the reach of the patent\ldots to a
particular technological use'' does not render an abstract idea patentable.
\sentence{diehr at 192 n. 14}. Nor does attachment of
``post-solution activity,''
\clause{flook at 584},
or
recitation of ``well-understood, routine, conventional activity previously
engaged by researchers in the field,''
\clause{mayo at 1300}.

In view of this clear precedent, the Court should hold that
attachment of a general-purpose computer does not render an abstract idea
patentable, in the present claims or otherwise.
Implementing an
abstract idea in the form of an algorithm on a general-purpose computer is a
``well-understood, routine, conventional activity,'' \clause{mayo at 1300}, that
merely
applies the algorithm in a ``particular technological environment,''
\clause{bilski at 3230 (quoting \sentence{diehr at 191-192})}.
Any ``competent draftsman'' could append elements of a
general-purpose computer to any algorithm.
This case is distinguishable from \inline{diehr}, which found patentable an
algorithm intimately tied to a specialized device, namely a rubber-curing
machine, \clause{see * at 187}, because unlike a rubber-curing machine, a
computer is able to perform any possible algorithm or mathematical
procedure.
Thus, in the claims at issue, the recitation of a general-purpose computer
should not render the claims eligible under \inline{101}.\footnote{The
contention that general-purpose computers become ``special-purpose computers''
when executing particular software is addressed in \inline{this at sec2c}.}



%%%%%%%%%%%%%%%%%%%%%%%%%%%%%%%%%%%%%%%%%%%%%%%%%%%%%%%%%%%%%%%%%%%%%%%%
%
% SECTION II
%
\section{The Court Should Proactively Clarify the Law of Subject Matter
Eligibility in Order to Avoid Further Errors Relating to Abstract Ideas}

The Supreme Court has taken numerous subject matter eligibility cases recently.
It does so because the Federal Circuit is in a confused state about the law of
\inline{101}, primarily because a small faction of that court repeatedly applies
incorrect analytical techniques to improperly find patents eligible even when
this Court's precedents demand otherwise.

To clearly enunciate the law for the Federal Circuit and to prevent the need for
further appeals, this Court should explicitly reject those improper analytical
techniques, some of which have been catalogued below.



%%%%%%%%%%%%%%%%%%%%%%%%%%%%%%%%%%%%%%%%%%%%%%%%%%%%%%%%%%%%%%%%%%%%%%%%
%
% SECTION II.A
%
\subsection{The Court Should Enunciate the Inappropriateness of Using
Specification Details to Evaluate Subject Matter Eligibility}

In assessing whether a claim is ineligible under \inline{101}, courts must
consider the entire breadth of the claim. Claims directed to an abstract idea
will still cover specific, concrete implementations of that abstract idea, so
the mere fact that a claim covers a concrete implementation is no indicator that
a claim is directed to eligible subject matter.

Nevertheless, certain judges of the Federal Circuit persistently err by relying
on specific examples to find patent claims eligible. In the present case, the
plurality opinion justified its finding that the system claims of the patents at
issue were eligible, by selecting a complex-looking flowchart from the
specification to point to the supposed complexity and concreteness of the claim.
By doing so, they failed to contemplate the possibility that other, simpler,
abstract ideas were \emph{also} covered by that same claim---ideas such as the
14-line computer program presented in this brief.

Ironically, those same judges of the Federal Circuit criticize their opposed
colleagues for failing to read the ``claims as a whole.'' It is in fact those
opposed colleagues who have actually read the claims as a whole, contemplating
the vast scope of what they cover. It is that plurality of the Federal Circuit,
instead, who fails to read the claims as a whole, focusing wrongly on specific
examples and obfuscatory language that misleadingly make abstract ideas appear
patentable.



%%%%%%%%%%%%%%%%%%%%%%%%%%%%%%%%%%%%%%%%%%%%%%%%%%%%%%%%%%%%%%%%%%%%%%%%
%
% SECTION II.B
%
\subsection{System Claims Are No More Eligible for Patenting than Method Claims}
\bblabel[Section \#]{sec2b}

The formalistic approach favored by some judges of the Federal Circuit lends to
easy circumvention by clever patent drafting. For example, the suggestion that
the method claims in the present case are ineligibile, while system claims
directed to the same technology are eligible, simply encourages patent
applicants to use system claims in order to skirt the abstract ideas test.

Granting such weight to mere formal drafting practices ignores the basic
rationale behind the Supreme Court's exceptions to \inline{101}. In explaining
the basis for the three exceptions to \inline{101}, this Court has applied the
fundamental principle that patents must ultimately incentivize innovation. While
patents on many inventions do serve this principle, patents to abstract ideas,
laws of nature and physical phenomena would in fact deter innovation by taking
away those ``basic tools of research available to all.''

Several judges of the Federal Circuit ignore this basic rationale. Judge Rader,
for example, has intimated that the three exceptions to \inline{101} are
essentially tautological, because one ``cannot invent an abstract idea, law of
nature or physical phenomenon'' since they have been around the whole time.

This unduly narrow, formalistic view of the exceptions to \S~101 fails to
adequately protect the concerns about incentives for innovation explicitly
relied upon by this Court. Under Judge Rader's view, mere addition of even the
most insignificant step to an otherwise abstract method would suddenly make that
abstract method patentable, because the combination would not have existed
before. The Court has specifically denounced this possibility, in holding
numerous times that insignificant post-solution activity and pre-solution
activity cannot render an otherwise abstract idea patentable.

\sentence{see johnson at 1077 (``\emph{Benson} applies equally whether an
invention is claimed as an apparatus or process, because the form of the claim
is often an exercise in drafting.'') (quoted in \sentence{alappat at 1542})}.



%%%%%%%%%%%%%%%%%%%%%%%%%%%%%%%%%%%%%%%%%%%%%%%%%%%%%%%%%%%%%%%%%%%%%%%%
%
% SECTION II.C
%
\subsection{Recitation of Details of a General-Purpose Computer Are Irrelevant
for Patent Eligibility}
\bblabel[Section \#]{sec2c}

This Court should make clear that a clever draftsman cannot turn an abstract
idea into patentable subject matter simply by reciting
aspects of a general-purpose computer, regardless of the level of detail with
which the claims describe the general-purpose computer. Several of the opinions
below were unduly impressed by detailed, technical language that in fact recited
nothing more than parts of a general-purpose computer, \clause{this at
computer-quotes}, and the Court should seek to counteract that position.

Consider a hypothetical ineligible claim to a method of performing long
division using pencil and paper, as explained above. It would be possible to
recite at length the physical attributes of the pencil and paper (``a pencil
comprising a wooden shaft surrounding a cylindrical graphite barrel, the wooden
shaft having a distal end including a rubber eraser, etc.''). But such a
recitation would affect neither the tendency of such a claim to effectively
preempt use of an abstract idea, nor the ineligibility of the claim. Allowing
patent eligibility to turn on this sort of insignificant detail ``would make the
determination of patentable subject matter depend simply on the draftsman's
art,'' a result that the Court should seek to avoid. \sentence{flook at 593}.

Just as recitation of detail about a pencil should not confer patent
eligibility, neither should recitation of detail about a general-purpose
computer.
Thus, language from the claims at issue, such as
``data storage unit'' and ``communications controller,'' should not affect the
ineligibility of the claims. The Court should reaffirm this point clearly.

One reason that the lower courts make this error is that they place undue
reliance on the 1994 Federal Circuit decision \inline{alappat}. In that case,
the lower court stated in dicta
that ``a general purpose computer in effect becomes a special purpose computer
once it is programmed'' with software. \sentence{* at
1545},\footnote{\inline{alappat} itself did not involve a general purpose
computer, but rather a special form of oscilloscope. \sentence{* at 1537}.}
Courts have used this statement to support a mistaken conclusion that recitation
of general-purpose computer hardware can confer patent eligibility.
\sentence{see, e.g., cls-enbanc at 1305 (Rader, J.); ultramercial at 1353}.

This reliance on \inline{alappat} is mistaken because the
Court's precedent regarding the abstract idea exception to \inline{101} does not
turn on whether a computer is labeled ``general purpose'' or ``special
purpose.''\footnote{As Judge Archer noted in dissent in \inline{alappat}, a
compact disc becomes special-purpose when music is recorded on it, but no patent
should issue on such a ``special-purpose compact disc.'' \sentence{* at
1553-1554}.} It turns on whether a patented claim would preempt virtually all
implementations of an idea, suppressing innovation along the way.
The Court should thus reject the
continued reliance of the lower courts on this dicta from \inline{alappat}.

By ensuring that patent eligibility does not turn on formal drafting practices,
such as recitation of system-style claims or inclusion of details of
general-purpose computer hardware, the Court will take \inline{101} analysis
from the metaphysical confusion that the lower courts have created, and return
it to first principles. At the core of those first principles, which date back
to the drafting of the Constitution, is the imperative
that the toolbox of abstract ideas must remain available to all. It is these
principles that
should guide the Court's decision.

\clearpage
\part{Conclusion}

For the foregoing reasons, \amici respectfully submit that the Court
should affirm the district court.

\signature

\appendix{Implementation of Claim 26 of the '375 Patent in \Numlines Lines of
Computer Code}
\bblabel[Appendix \#]{appx:code}
\appendixpagenumbers

The following \numlines-line computer program, written in the \textsc{basic}
programming language, implements Claim 26 of the '375 Patent.\footnote{A
\textsc{basic} program interpreter to run this program is available at
http://www.vintage-basic.net/.}

\vskip\baselineskip
\inlinebox{\wholeprogram}
\vskip\baselineskip

The subsequent text reviews the elements of the claim in detail and explains how
a general-purpose computer, running the above computer program, would satisfy
all the elements of the claim. For convenience, the entirety of the claim is
reprinted in the next appendix.

\claimbox{preamble}{\claimelt0}

The preamble recites that the claim covers a general purpose computing system,
called a ``data processing system'' by the claim language. The recitation that
the system is ``to enable the exchange of an obligation'' is a statement of
field of use or intended use, which should not contribute to the scope of the
claim.
\sentence{see bilski at 3231 (``[L]imiting an abstract idea to one field of
use\ldots did not make the concept patentable.''); mpep at S 2103/IC
(instructing that ``statements of intended use or field of use'' ``may raise a
question as to the limiting effect of the language in a claim'')}.

\claimbox{elements 1--2}{\claimelt1\par\claimelt2}

These elements recite general hardware inherent in a general purpose computer. A
``communications controller'' broadly refers to a component of a computer that
receives and processes communications, and a ``first party device'' could refer
to any computer hardware.\footnote{Petitioner has at least once
described the communications controller as a device ``that allows communications
over a wide-area network.'' \sentence{cls-dist95 at 6}. But the text of the
patent belies that limited definition. \sentence{see pat375 at column 7, lines
46-57 (``A number of communications controllers\ldots effect communications
between the processing units and various external hardware devices\ldots. A
large range of communications hardware products are supported, and collectively
are referred to as the stakeholder input/output devices.'' (reference numbers
omitted))}.} A computer must communicate with its users in order to be useful,
so these components are necessary to any computer.

\codeandclaimbox{elements 3--5}{\claimelt3\par\claimelt4\par\claimelt5}{1}{2}

These elements of the claim simply require that a computer store two numbers
representing account balances. The ``data storage unit'' might be any computer
storage component, such as a hard disk or memory. The ``information about'' the
first and third accounts broadly encompass any account information, such as an
account balance.\looseness=1

The recitations that the information be stored ``independent from'' various
accounts maintained by exchange institutions are simply statements of intended
use, which should not contribute to the patentability of the claim. Petitioners
have never suggested that the external exchange institutions are necessary
parties to infringement of their claims. Furthermore, so long as the two stored
numbers reflect actual account balances in external banks, the ``independent
from'' limitations are satisfied.

The computer code implements these elements of the claim by instructing a
computer to store two account balances, into variables named \emph{account1} and
\emph{account3}.

\claimbox{element 6}{\claimelt6}

This element is simply further recitation of details about a general purpose
computer. Any computer would necessarily be coupled to a data storage unit, so
that it might access data for processing, and further be coupled to a
communications controller, so that it may receive and output information.

\codeandclaimbox{element 7}{\claimelt7}{3}{3}

According to this element, the computer receives a ``transaction.'' An exchange
of money between two accounts is one type of transaction, and Petitioners have
used described an ``exchange'' as an example of a transaction. (Petr.'s Br.\ 7.)
Thus, this
element requires nothing more than receipt of an instruction to transfer money
between two accounts.

The computer code implements this element by requesting the user to input an
amount of money to
transfer between the first and third account. This is performed by the
\textsc{input} command. Upon running this line of code,
a computer would print out the prompt message, and then await an outside user to
enter a number indicating the amount of money to transfer. The amount to
exchange is stored in a variable named \emph{exchange}.

\codeandclaimbox{element 8}{\claimelt8}{4}{6}

This element describes two operations. First, a computer must check that at
least one of the accounts has a large enough balance to permit the desired
transfer of money (``ensuring that said first party\ldots ha[s] adequate value
in
said first account''). Second, the computer must record the transfer by
adjusting the balances of the accounts (``electronically adjust said first
account and said third account'').

Note the substantial presence of inoperative language in this claim element. The
recitation ``in order to effect an exchange obligation arising from said
transaction between said first party and said second party'' does nothing more
than reiterate that the computer is transferring money between accounts.
Furthermore, the claim recites that the computer must ensure ``adequate value in
said first account and/or said third account,'' and the disjunctive ``and/or''
means that the claim element is satisfied if only one of those accounts is
checked. \sentence{see mpep at S 2103/IC (``Language that suggests or makes
optional but does not require steps to be performed\ldots does not limit the
scope of a claim or claim limitation.'')}.

The computer code implements the step of checking the account balances at line
40, which halts execution (with \textsc{stop}) if the balance of
\emph{account1} is less than the amount to be exchanged. The code implements the
step of effecting the transfer at lines 50--60, which deducts the amount to be
exchanged from \emph{account1} and adds that amount to \emph{account3}.

\codeandclaimbox{element 9}{\claimelt9}{7}{7}

This claim element requires only that a computer display an instruction to
perform the desired transfer of money. The claim element recites ``an
instruction to said first exchange institution and/or said second exchange
institution,'' but the disjunctive ``and/or'' means that a single instruction
suffices. Similarly, the recitation of an instruction ``to adjust said second
account and/or said fourth account'' only requires an instruction with regard to
a single account.

The requirement that the instruction be ``an irrevocable, time invariant
obligation'' is merely a statement of intended use that should not contribute to
the patentability of the claim. An instruction is simply a text, and the
recipient of the instruction chooses whether to treat that text as irrevocable
or time-invariant. Although this claim language could plausibly have been
defined in the specification to require some sort of special format for the
instruction, Petitioners have never identified any such special definition in
any of their briefs to this Court, the Federal Circuit, or the district
court,\footnote{The district court briefs reviewed are identified on the docket
as Documents Nos.\ 53, 54, 68, 95, and 99. The Federal Circuit briefs reviewed
are identified on the docket as Documents Nos.\ 22, 33, 41, and 194.}
and the text of the
specification contains neither term outside of the claims. Furthermore, even if
these terms did have some special meaning, it would only dictate the content of
the instruction text, and content of text does not contribute to patentability.

The computer code implements this element by causing a computer to print an
instruction to adjust the second account. The instruction directs the
first institution to deduct the amount \emph{exchange} from the account.

\appendix{Claim 26 of the '375 Patent}
\bblabel[Appendix \#]{appx:claim}
\addtotoa{pat375}

\emph{Numbers, in square brackets, have been inserted before each element of the
claim, to assist in referring to claim elements within the brief.}

\wholeclaim

\end{document}

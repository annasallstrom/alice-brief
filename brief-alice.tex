\documentclass{scotus}
\usepackage{bluebook}
\usepackage{ifxetex}
\usepackage{xspace}
\usepackage{enumitem}
\setlist{topsep=0pt,partopsep=0pt,parsep=0pt,itemsep=\parskip}

\usepackage{listings}
%
% This is a really awful hack that causes the listings package to output double
% quotation marks for strings with proper directional quotation marks.
%
\makeatletter
\def\lst@Append#1{%
    \advance\lst@length\@ne
    \if\noexpand#1"%
        \lst@hack@quote
    \else
        \lst@token=\expandafter{\the\lst@token#1}%
    \fi
}
\newif\if@lst@hack@inquote
\def\lst@hack@quote{%
    \if@lst@hack@inquote
        \global\@lst@hack@inquotefalse
        \lst@token=\expandafter{\the\lst@token''}%
    \else
        \global\@lst@hack@inquotetrue
        \lst@token=\expandafter{\the\lst@token``}%
    \fi
}
\makeatother


\defcase{i4i}{
p=Microsoft Corp.,
d=i4i Limited Partnership,
cite=131/S. Ct./2238,
year=2011
}

\defcase{bilski}{
p=Bilski,
d=Kappos,
cite=130 S. Ct. 3218,
year=2010
}

\defstatcode{112p2}{
cite={35/U.S.C./S 112, \P~2},
year=2011
}

\defcase{quanta}{
p={Quanta Computer, Inc.},
d=LG Electronics Corp.,
cite=553/U.S./617,
year=2008
}

\defstatcode{101}{
cite=35 U.S.C. S 101,
year=2013
}

\defcase{flook}{
p=Parker,
d=Flook,
cite=437 U.S. 584,
year=1978,
inlinedefendant
}

\defcase{funkbros}{
p=Funk Brothers Seed Co.,
d=Kalo Inoculant Co.,
cite=333 U.S. 127,
year=1948
}

\defcase{leroy}{
p=Le Roy,
d=Tatham,
year=1852,
cite=55 U.S. (14 How.) 156
}

\defcase{benson}{
p=Gottschalk,
d=Benson,
inlinedefendant,
cite=409 U.S. 63,
year=1972
}

\defcase{mayo}{
p=Mayo Collaborative Services,
d={Prometheus Laboratories, Inc.},
cite=132 S. Ct. 1289,
inline=Mayo,
year=2012
}

\defcase{cls-enbanc}{
p=CLS Bank International,
d=Alice Corp.,
cite=717 F.3d 1269,
year=2013,
court=Fed. Cir.,
enbanc
}

\defpatent{pat375}{
number={7725375},
date={June 27, 2005}
}

\ExplanatoryPhrase{vacated and rehearing en banc granted,}

\defcase{cls-panel}{
p=CLS Bank International,
d=Alice Corp.,
cite=685 F.3d 1341,
year=2012,
court=Fed. Cir.,
paren=panel decision,
subsequent={vacated and rehearing en banc granted, cls-vacate}
}
\defcase{cls-vacate}{
sameparties=cls-panel,
cite=484 Fed. Appx. 559,
year=2012,
court=Fed. Cir.
}

\defcase{diehr}{
p=Diamond,
d=Diehr,
cite=450 U.S. 175,
inlinedefendant,
year=1981
}
\defbook{mpep}{
    title=Manual of Patent Examining Procedure,
    instauth=United States Patent and Trademark Office,
    year=2012,
    edition={8th ed., 9th rev.},
    hereinafter=MPEP
}

\defcasedoc{cls-dist95}{
sameparties=cls-panel,
docket=1:07-cv-974,
cite=768 F. Supp. 2d 221,
docname=Alice Corp. Pty. Ltd.'s Renewed Cross-Motion for Partial Summary
Judgment as to Subject Matter Eligibility,
date={Sept. 22, 2010},
court=Dist. D.C.,
paren=Doc.\ No.\ 95
}
\defcasedoc{cls-dist53}{
sameparties=cls-panel,
docket=1:07-cv-974,
cite=768 F. Supp. 2d 221,
court=Dist. D.C.,
docname=Alice Corp. Pty. Ltd.'s Cross-Motion for Partial Summary
Judgment as to Subject Matter Eligibility,
date={Apr. 3, 2009},
paren=Doc.\ No.\ 53
}
\defcasedoc{cls-dist54}{
sameparties=cls-panel,
docket=1:07-cv-974,
cite=768 F. Supp. 2d 221,
docname=Alice Corp. Pty. Ltd.'s Cross-Motion for Partial Summary
Judgment as to Subject Matter Eligibility,
date={Apr. 3, 2009},
court=Dist. D.C.,
paren=Doc.\ No.\ 54
}
\defcasedoc{cls-dist68}{
sameparties=cls-panel,
docket=1:07-cv-974,
cite=768 F. Supp. 2d 221,
court=Dist. D.C.,
docname=Alice Corp. Pty. Ltd.'s Reply Memorandum in Support of its Cross-Motion
for Partial Summary Judgment as to Patent-Eligibility,
date={May 19, 2009},
paren=Doc.\ No.\ 68
}
\defcasedoc{cls-dist99}{
sameparties=cls-panel,
docket=1:07-cv-974,
cite=768 F. Supp. 2d 221,
court=Dist. D.C.,
docname=Alice Corp. Pty. Ltd.'s Reply Memorandum in Support of its Renewed
Cross-Motion for Partial Summary Judgment as to Patent-Eligibility,
date={Oct. 22, 2010},
paren=Doc.\ No.\ 99
}
\defcasedoc{cls-app22}{%
sameparties=cls-panel,
docket=11-1301,
cite=685 F.3d 1341,
court=Fed. Cir.,
docname=Brief of Defendant-Appellant Alice Corp. Pty. Ltd.,
date={June 24, 2011},
paren=Doc.\ No.\ 22
}
\defcasedoc{cls-app33}{%
sameparties=cls-panel,
docket=11-1301,
cite=685 F.3d 1341,
court=Fed. Cir.,
docname=Reply Brief of Defendant-Appellant Alice Corp. Pty. Ltd.,
date={Sept. 16, 2011},
paren=Doc.\ No.\ 33
}
\defcasedoc{cls-app194}{
sameparties=cls-panel,
docket=11-1301,
cite=717 F.3d 1269,
court=Fed. Cir.,
docname=\emph{En Banc} Response Brief of Defendant-Appellant Alice Corp. Pty.
Ltd.,
date={Jan. 10, 2013},
paren=Doc.\ No.\ 194
}

\newcounter{claimcounter}
\def\theclaim{%
A data processing system to enable the exchange of an obligation
between parties, the system comprising:
  \element a communications controller,
  \element a first party device, coupled to said communications controller,
  \element a data storage unit having stored therein
    \element\sub (a)\ind information about a first account for a first party,
    independent from a second account maintained by a first exchange
    institution, and
    \element\sub (b)\ind information about a third account for a second party,
    independent from a fourth account maintained by a second exchange
    institution; and
  \element a computer, coupled to said data storage unit and said
  communications controller, that is configured to
    \element\sub (a)\ind receive a transaction from said first party device via
    said communications controller;
    \element\sub (b)\ind electronically adjust said first account and said
    third account in order to effect an exchange obligation arising from said
    transaction between said first party and said second party after ensuring
    that said first party and/or said second party have adequate value in said
    first account and/or said third account, respectively; and
    \element\sub (c)\ind generate an instruction to said first exchange
    institution and/or said second exchange institution to adjust said second
    account and/or said fourth account in accordance with the adjustment of said
    first account and/or said third account, wherein said instruction being an
    irrevocable, time invariant obligation placed on said first exchange
    institution and/or said second exchange institution.
}
\newcount\claimeltcount
\def\claimelt#1{%
    \begingroup
        \let\element\or
        \let\sub\relax
        \let\preamble\relax
        \let\ind\space
        \claimeltcount=#1\relax
        \expandafter\ifcase\expandafter\claimeltcount\theclaim\fi
        \unskip
    \endgroup
}
\def\wholeclaim{%
    \begingroup
        \let\preamble\par
        \def\element{\par\endgroup\begingroup\advance\leftskip\parindent}%
        \def\sub{\advance\leftskip\parindent}%
        \let\ind\quad
        \begingroup\theclaim\par\endgroup
    \endgroup
}


% This sets up various macros used in this brief.
\setcounter{passimnum}{5}

%
% If you do not have these fonts, comment the next few lines out by changing
% \ifxetex to \iffalse
%
\ifxetex
    \usepackage{fontspec}
    \setmainfont[BoldFont=ITC Century Std-Bold,Ligatures=TeX]{%
        Century Expanded LT Std%
    }
    \def\titlefont{\fontspec{Cloister Black}\LARGE}
    \def\basickeywordstyle{\addfontfeatures{Letters=UppercaseSmallCaps}}
\else
    \def\basickeywordstyle{\bfseries}
\fi

%
% Macros for how long the computer program is
%
\def\numlines{seven\xspace}
\def\Numlines{Seven\xspace}
\def\wholeprogram{\programlines{1}{7}}

%
% Macros to format the claim language and/or computer program
%
\def\programlines#1#2{%
    \lstset{keywordstyle=\basickeywordstyle,identifierstyle=\itshape}
    \lstset{breaklines=true,showstringspaces=false}
    \lstset{aboveskip=0pt,belowskip=0pt}
    \lstset{language=[Visual]Basic,columns=fullflexible}
    \lstinputlisting[firstline=#1,lastline=#2]{%
        implementations/375claim26.bas%
    }%
}
\long\def\inlinebox#1{%
    \vskip\parskip
    \hrule\kern1.5pt \hrule\nobreak
    \begingroup
        \vskip 4pt
        \parskip=0pt
        #1\par
        \nobreak\vskip 4pt
    \endgroup
    \hrule\kern1.5pt \hrule
}
\long\def\claimtext#1#2{%
    \noindent
    \bblabel{appx:claim:#1}%
    \textsc{Claim 26, #1:}\par
    {\itshape#2\unskip}%
}
\long\def\claimbox#1#2{%
    \vskip.5\baselineskip
    \inlinebox{%
        \claimtext{#1}{#2}%
    }%
}
\long\def\codeandclaimbox#1#2#3#4{%
    \vskip.5\baselineskip
    \inlinebox{%
        \claimtext{#1}{#2}%
        \vskip 4pt
        \hrule
        \vskip 4pt
        \noindent
        \ifnum#3=#4\relax
            \textsc{Computer code, line #30:}\par
        \else
            \textsc{Computer code, lines #30--#40:}\par
        \fi
        \programlines{#3}{#4}%
    }%
}%

%
% Other useful macros
%
\def\Amici{\emph{Amici}\xspace}
\def\amici{\emph{amici}\xspace}

\endinput


\docket{13-298}
\casecaption{
\textsc{Alice Corporation Pty. Ltd.},\\\The{Petitioner,}\\v.\\
\textsc{CLS Bank International and CLS Services Ltd.},\\
\The{Respondent.}
}

\posture{On Writ of Certiorari \\
to the United States Court of Appeals \\
for the Federal Circuit}

\counsel{
\textsc{Charles Duan} \\
\counselofrecord \\
\textsc{Public Knowledge} \\
1818 N St NW, Suite 410 \\
Washington, DC 20036 \\
(202) 861-0020 \\
cduan@publicknowledge.org \\
\and
\textsc{Jack Lerner} \\
\textsc{USC Intellectual Property and Technology Law Clinic} \\
699 Exposition Blvd, Room 425 \\
Los Angeles, CA 90089 \\
(213) 740-2537 \\
iptlcsct@law.usc.edu
}
\counselis{Counsel for Amici Curiae}

\title{Brief of Public Knowledge \protect\\
and the Application Developers Alliance \protect\\
as \emph{Amici Curiae} in Support of Respondent}

\begin{document}

\maketitle

\romanpagenumbers
\tableofcontents

\part{Table of Authorities}

\tableofauthorities

\clearpage
\arabicpagenumbers

\part{Interest of \emph{Amici Curiae}}

Public Knowledge is a non-profit organization that is dedicated to preserving
the openness of the Internet and the public's access to knowledge; promoting
creativity through balanced intellectual property rights; and upholding and
protecting the rights of consumers to use innovative technology lawfully. As
part of this mission, Public Knowledge advocates on behalf of the public
interest for a balanced patent system, particularly with respect to new and
emerging technologies.\footnote{Per Supreme Court Rule 37(6), no counsel for a
party authored this brief in whole or in part, and no counsel or party made a
monetary contribution intended to fund the preparation or submission of the
brief. No person or entity, other than \amici, their members, or their counsel,
made a monetary contribution to the preparation or submission of this brief. Per
Rule 37(3)(a), consent has been granted for the filing of this brief, as
indicated by the blanket consents from counsel for petitioner and counsel for
respondents docketed December 11, 2013.}

Public Knowledge has previously served as \emph{amicus} in key patent cases.
\sentence{e.g., i4i; bilski; quanta}.

The Application Developers Alliance (ADA) is a nonprofit industry association
comprising more than 30,000 individual software developers and more than 150
companies who design and build applications (“apps”) for consumers to use on
mobile devices like smartphones and tablets. Apps run on software platforms,
including Google’s Android, Apple’s iOS, and Facebook, and are sold or
distributed through virtual stores like Google’s Play Store. ADA is dedicated to
meeting the needs of app developers as creators, innovators, and entrepreneurs,
by delivering essential information and resources and by advocating for public
policies that promote the app ecosystem.

App developers are both central to innovation and vulnerable to the patent laws
that surround innovation. By innovating rapidly and cheaply, app developers
represent an increasingly robust force in the economy. The app economy is now
globally valued at over \$53 billion and has
created approximately 466,000 jobs in the United States since
2007.\footnote{\sentence{pappas; mandel at 13}.} But many app developers,
including
ADA members, are struggling as a result of abusive patent assertion, especially
that originating from patent assertion entities (PAEs). Such entities often
assert overly broad patents, propounding unfounded infringement allegations and
aggressive litigation threats, which deeply chill
innovation.\footnote{\sentence{wildtangent-ada}.}

Inconsistency and uncertainty in areas of patent law, such as subject matter
eligibility, are enabling factors in PAE litigation, as they enable aggressive
patent assertors to take improper, overbroad positions.
\sentence{e.g., eon-net at 1326-1328}. This forces many
app developers to conclude that innovation is not worth the expensive baggage of
defending against such claims, resulting in delays to and deficiencies in app
development and overall innovation.\footnote{\sentence{see, e.g., chien-oti at
17}.} Thus,
ADA and its members have a strong interest in this Court providing clarity in
this area of patent law.



\fullcite{i4i}
\fullcite{bilski}
\fullcite{quanta}

\clearpage

\part{Summary of Argument}

Abstract ideas are not eligible for patenting because, as this Court has
steadfastly maintained, certain fundamental subject matter must be fixed in the
public domain, so that patents may serve their constitutional mandate to
``promote the Progress of Science and the useful Arts.'' Being the basic tools
of innovation, abstract ideas must remain available to the public; to do
otherwise would impede innovation more than promote it.

This case tests how far a patent may encroach on that valuable domain
reserved to innovators, creators, and the public. Petitioner holds patents
to computer technology. The patent claims at issue are
lengthy and detailed, some over two hundred words long. But those claims
actually cover very simple ideas; the verbose language is a mere facade
masking basic concepts.

To demonstrate this, \amici have implemented one of those 200-word
claims---in only 7 lines of computer code.

This computer program implementation shows that the patent
claims are directed to nothing more than an abstract idea implemented on a
general-purpose computer, which should not be patent-eligible.
%
% Summary of I.C argument
To hold otherwise would contravene the
Court's precedent and undermine the rationale for unpatentability of
abstract ideas. Such ``abstract-idea-plus-computer'' patents would be effective
monopolies on the basic tools of innovation, a result that the
Court has adamantly rejected.

To prevent further errors of this sort, \amici identify three points of
clarification on the law of subject matter eligibility, and urge the Court to
enunciate these specific points. Doing so not
only will correct the judgment below and guide the lower courts, but also will
ensure that those valuable basic tools of innovation remain available to all.

\part{Argument}

This case presents the recurrent question of what constitutes patentable
subject matter, particularly with regard to the fields of computer software and
business methods. \Amici address two aspects of this question as they relate to
the present case. First, \amici show that the patented claims at issue are
directed to ineligible abstract ideas, by
implementing one of those claims is used to assist in this demonstration.
Second, in view of the fractured opinions of the Federal Circuit below, \amici
suggest three principles for guiding the lower courts in deciding future cases.



%%%%%%%%%%%%%%%%%%%%%%%%%%%%%%%%%%%%%%%%%%%%%%%%%%%%%%%%%%%%%%%%%%%%%%%%
%
% SECTION I
%
\section{The Claims at Issue Are Ineligible for Patenting Because They
Preempt an Abstract Idea}

The question presented is whether Petitioner's claims are directed to
patent-eligible subject matter. Generally, ``any new and
useful process, machine, manufacture, or composition of matter'' is eligible for
patenting. \sentence{101}. But three exceptional fields
are nevertheless ineligible: laws of nature, physical phenomena, and abstract
ideas. \sentence{e.g., mayo at 1293 (quoting \sentence{diehr at 185})}.

The claims of these patents, as with many patents in the computer technology
field, are full of complex technical language.
But these claims actually present very basic concepts---so basic, in fact, that
\amici have prepared a \numlines-line computer program
that implements all the features of one of the most complex claims. The program,
shown in Figure~\ref{code-listing} on page~\pageref{code-listing},
demonstrates that the claims recite not specialized, technical systems, but
rather a broad, general, simple algorithm that reduces to nothing more than an
abstract idea run on a computer. Because mere application of a general-purpose
computer should not render an otherwise abstract idea patentable, \amici
urge the Court to find the present claims ineligible.



%%%%%%%%%%%%%%%%%%%%%%%%%%%%%%%%%%%%%%%%%%%%%%%%%%%%%%%%%%%%%%%%%%%%%%%%
%
% SECTION I.A
%
\subsection{The Claims Can Be Implemented in Just \Numlines Lines of
Computer Code}

Much of the disagreement in the lower court's fractured decision stemmed
from a disagreement over the nature of the patent claims at issue. Judge Lourie,
writing for five judges,
found the claims to recite ``a handful of
computer components in generic, functional terms that would encompass any
device'' and unduly preempt an abstract idea. \sentence{see cls-enbanc at 1290}.
Judge Rader, writing for four judges, found those same claims narrowly tailored,
``limited to an implementation that includes at least four separate structural
components'' rendering the claim patent-eligible. \sentence{see cls-enbanc at
1307}.

The claims at issue do use technical-sounding, complex language, making them
appear to be directed to a narrowly tailored invention. One of the claims at
issue recites, among other things, a ``communications controller,'' a
``data storage unit,'' and an ``instruction being an irrevocable, time invariant
obligation.'' \sentence{pat375 at claim 26, col. 66-67}.\footnote{Claim 26 of
\inline{pat375} is considered in this brief because it was found patentable by
the greatest number of judges of the lower court decision. \sentence{see
cls-enbanc at 1309 (Rader, Linn, Moore \& O'Malley, JJ.);
cls-enbanc at 1327 (Newman, J.)}.
\Amici could have easily used any other claim at issue.
For reference, Claim
26 is reprinted in \inline{this at appx:claim}.}

But beneath this veneer
of technical language is nothing more than a very simple, basic idea. As a
demonstration, the computer program shown in Figure~\ref{code-listing}
on the opposite page,
implements all the features of Claim 26 of \inline{pat375}.
A complete explanation of the working of
this program as it relates to Claim 26 of \inline{pat375} is presented in
\inline{this at appx:code}.

\def\floatpagefraction{.1}
\begin{figure}[p]
\inlinebox{\wholeprogram}%
\vskip2\baselineskip
\caption{Implementation of Claim 26 of \protect\inline{pat375}.}
\label{code-listing}
\end{figure}

As the Court will observe, the computer program is only \numlines lines long,
indicating that
the verbose language of the claims does not in fact demand specific, particular
implementations but rather expansively preempts all uses of a simple, basic
idea. A \numlines-line computer program is remarkably simple in comparison to
ordinary computer programming:
\begin{itemize}
\item A single page of the Supreme Court's website is 926 lines long, including
145 lines of computer code.\footnote{\sentence{scotus-web}.}
\item A fourteen-year-old wrote an iPhone app with over 11,000 lines of
code.\footnote{\sentence{forbes-iphone}.}
\item The computer program that formatted the citations
and table of authorities of this brief is 7,939 lines long.\footnote{That
program, which was written by counsel of record on this brief, is available at
\url{https://github.com/charlesduan/alice-brief}.}
\end{itemize}

Certain
\bblabel{computer-quotes}
judges below were misled by the language of the
claims and the patent.
Judge Rader believed that Claim 26 involved
``a complex problem'' that could only be solved with a specialized system
with ``at least four separate structural components.'' \sentence{cls-enbanc at
1307}. He reviewed the ``at least thirty two figures
which
provide detailed algorithms'' to conclude that ``[l]abeling
this system claim an `abstract concept' wrenches all meaning from those words.''
\sentence{cls-enbanc at 1309}. Judge Moore likewise found a
similarly-worded claim ``limited to one that is configured to perform certain
functions in a particular fashion'' and, based on one of the
flowcharts, suggested that the claims demanded a dizzyingly long
and complex algorithm. \sentence{cls-enbanc at 1318}. And Judge Linn
concluded that, while they may be based
on an abstract idea, ``the claims here are directed to very specific ways of
doing that.''
\sentence{cls-enbanc at 1741}.

The common thread among all of these opinions is an assumption that, given the
heavy use of technical language in the specification and claims, only a
specific, complex, technical computer program could infringe the patents. As the
above \numlines-line computer program demonstrates, this assumption was in
error.

The computer program devised by \amici reads
the claim as a whole, as this Court requires. \sentence{see diehr at 188}. As
the detailed appendix shows, every
claim limitation is considered and implemented appropriately in the
computer code, so it cannot be said that details or limitations have been
stripped from the claim. \sentence{see this at appx:code; cf. diehr at 188 (``It
is inappropriate to dissect the claims into old and new elements and then to
ignore the presence of the old elements in the analysis.'')}. Furthermore,
because
the computer program is a functional, working implementation of the claim, it
cannot be argued that it is a mere abstraction or generalization of the claims.

Thus, Claim 26 of \inline{pat375} is directed not to a complex
system requiring specialized hardware, but rather to a basic,
\numlines-line computer algorithm.


%%%%%%%%%%%%%%%%%%%%%%%%%%%%%%%%%%%%%%%%%%%%%%%%%%%%%%%%%%%%%%%%%%%%%%%%
%
% SECTION I.B
%
\subsection{The Claims Cover All
Computer Implementations of an Abstract Idea}
\bblabel[Section \#]{sec1b}

The example computer program shows that the asserted claims, though lengthy and
technical in appearance, are actually directed only to a very simple, basic
computer procedure. The example computer program
shows that the asserted claims are directed only to the abstract idea of
accounting by a third-party escrow.

This Court's precedent lays out several guidelines for determining whether a
claim is directed to an abstract idea.
``A principle, in the abstract, is a fundamental truth; an original cause; a
motive; these cannot be patented.''
\sentence{benson at 67 (quoting \sentence{leroy at 175})}.
%Claims directed merely to an abstract idea are
%not eligible under \inline{101}, because they are the ``basic tools of
%scientific and technological work,'' \clause{benson at 67}, which therefore
%are ``part of the storehouse of knowledge of all men\ldots free to all men and
%reserved exclusively to none,'' \clause{bilski at 3225 (quoting
%\sentence{funkbros
%at 130}) (omission in original)}. Furthermore, inclusion
Furthermore, ``conventional or
obvious''
post-solution or pre-solution activity cannot
render a claim eligible, because otherwise
``a competent draftsman could attach some form of post-solution activity'' to
``transform an unpatentable principle into a patentable process.''
\sentence{flook at 590; see also mayo at 1300 (holding ``conventional steps,
specified at a high level of generality,'' to similarly not confer patent
eligibility)}.

According to these guidelines, it is clear
that every element of Claim 26 is (1) an inherent aspect of the
abstract idea of third-party escrow, (2) a conventional component of a
general-purpose computer, or
(3) insignificant pre- or post-solution activity.
The claim is therefore
ineligible.


Elements\footnote{This brief references elements by numbers corresponding to
the claim reprinted in the appendix. \sentence{see this at appx:claim}.} 1--2 of
the
claim describe ordinary components of a general-purpose computer. \sentence{see
this at appx:claim:elements 1--2}. ``Communications controller'' and
``first party device'' are broad, general terms that encompass basic computer
components for interacting with users.\footnote{\sentence{see turing at 231-232
(describing the Turing machine, a fundamental model for all computers, as
including a ``paper tape'' for communicating with the user)}.} Furthermore,
these two components are only recited in conjunction with a step of receiving
data, which as explained below is insignificant pre-solution activity.
\nofullcite{turing}

Elements 3--5 describe basic record-keeping operations inherent in the idea of
third-party escrow. Although the claim language verbosely describes a ``data
storage unit'' with ``information about a first account'' and second account,
the computer program demonstrates that these elements in fact
require nothing more than recording two
numbers in a computer. \sentence{see this at appx:claim:elements 3--5}.
Certainly one would necessarily store such account information as part of an
escrow service.\footnote{The ``data storage unit'' is an essential part of a
general-purpose computer. \sentence{see turing at 231-232 (further explaining
that the Turing machine includes an \emph{m}-configuration for storing the state
of the machine)}.}

Element 6 recites ``a computer,'' and as such only further describes a
general-purpose computer.

Element 7 states that the computer must ``receive a transaction.'' This Court
and others have repeatedly held that steps of
obtaining data to be used for processing
constitute insignificant pre-solution activity. \sentence{see, e.g., mayo at
1297-1298 (treating as pre-solution activity a step of determining a level of
metabolites prior to adjusting a treatment); meyer at 794 (``[A] data gathering
step\ldots cannot make an otherwise nonstatutory claim statutory.'')}. As such,
this
claim element does not contribute to the eligibility of the claim.

Element 8 describes two steps to be performed by the computer, both of which are
inherent in the idea of third-party escrow. First, the computer is tasked with
``ensuring that said first party and/or said second party have adequate value''
in their accounts. The computer code shows that this amounts to nothing more
than a comparison, checking whether an account balance is greater than an amount
to be transferred out of that account. \sentence{see this at appx:claim:element
8}. This is the basic purpose of a third-party escrow broker, who
must ensure that the parties' accounts contain sufficient funds.

Second, element 8 requires the computer to ``electronically adjust said first
account and said third account.'' This operation, which amounts to only two
lines of computer code,
is inherent in any third-party escrow service, which must adjust account balance
records to account for a transaction.

Element 9 instructs that the computer ``generate an instruction to said first
exchange institution and/or said second exchange institution to adjust said
second account and/or said fourth account.'' Despite the fifty-nine-word length
of this element, it reduces to a single operation: producing a
message describing the transaction just completed. \sentence{see this
at appx:claim:element 9}. This elementary output step is quintessential
post-solution activity
that should not contribute to the eligibility of the claim. \sentence{cf.
flook at 590 (treating as post-solution activity a step of adjusting an alarm
limit following a computation)}.

Claim 26 is directed to nothing more than an abstract idea of
third-party escrow, in conjunction with insignificant pre-solution and
post-solution activity, and ordinary---albeit verbosely described---components
of
a general purpose computer.
This Court should accordingly hold the claim, and all others like it,
unpatentable.



%%%%%%%%%%%%%%%%%%%%%%%%%%%%%%%%%%%%%%%%%%%%%%%%%%%%%%%%%%%%%%%%%%%%%%%%
%
% SECTION I.C
%
\subsection{Claims that Preempt Substantially All Computer Implementations of an
Abstract Idea Should Be Ineligible}
\bblabel[Section \#]{sec1c}

% Trying to shorten things
\iffalse
A patent claim with the ``practical effect'' of removing an abstract idea from
the public domain is ineligible under \inline{101}. \sentence{see benson at
71-72}.
Claim 26 of the '375 patent would have the practical effect of removing all uses
of an abstract idea \emph{implemented on a general-purpose computer} from the
public domain.
\fi

The Court should hold that the recitation of a
general-purpose computer, as in the claims at issue here, does not render the
claims eligible under \inline{101}.
This principle follows, first, from the goal of promoting innovation, a goal
central to
the Court's \inline{101} doctrine, and second, from the rules of law
the Court has derived from these principles.

The abstract ideas exception is grounded in the principle that certain
fundamental subject matter must be fixed in the public domain, so that
patents may serve their constitutional mandate to ``promote the Progress
of\ldots the useful Arts.'' \sentence{patentclause}.
Abstract ideas are unpatentable because they
are ``the basic tools of scientific and technological work,'' \clause{benson at
67}, and must remain ``free to all men and reserved exclusively to none,''
\clause{bilski at 3218 (quoting \sentence{funkbros at 130})}. ``[M]onopolization
of those tools through the grant of a patent might tend to impede innovation
more than it would tend to promote it.'' \sentence{mayo at 1293; accord bilski
at 3228 (patent law must avoid ``granting
monopolies over procedures that others would discover by independent, creative
application of general principles'')}.

To permit patents on abstract ideas merely tied to general-purpose computers
would eviscerate this principle.
Computers are in
widespread use today, and they are essential to innovation and a productive
economy. \sentence{see, e.g., brynjolfsson at 4}. 
Allowing patents on abstract ideas merely tied to computers
would relegate
innovators to practicing abstract ideas on pencil and
paper. Needless to say, given the general importance of computers, such an
absurd state of affairs would severely hamstring innovation.
The basic tools of innovation must remain basic tools available to all, even
when they are, or must be, implemented on general-purpose technologies.

As an analogy, consider a patent claim directed to long division performed with
pencil and paper. Long division can, in theory, be practiced in the mind, but
as a practical matter no ordinary person can do so. Thus, this pencil-and-paper
patent would effectively make the abstract idea of long division unusable.
Similarly, computers are capable of tasks that ordinary
humans cannot perform unaided, even though those tasks may be
abstract ideas. The public must be able to apply these abstract ideas to
computers if those abstract ideas are to
remain ``free to all men and reserved exclusively to none.''\footnote{%
Advances in computer hardware are of course themselves eligible for patent
protection.  \sentence{see cls-enbanc at 1292 (Lourie, J., concurring)
(observing in dicta that computers \emph{per se} are ``surely patent-eligible
machines'')}. Equally so would be advances in pencil technology. But these are
distinguishable from mere annexation of abstract ideas to computers or pencils.}

In view of this important principle, this Court has eschewed
formalistic exegesis in favor of a practical analysis of the actual, effective
scope of the claims.
So, for example,
``limiting the reach of the patent\ldots to a
particular technological use'' does not render an abstract idea patentable.
\sentence{diehr at 192 n. 14}. Nor does attachment of
``post-solution activity,''
\clause{flook at 584},
or
recitation of ``well-understood, routine, conventional activity previously
engaged by researchers in the field,''
\clause{mayo at 1300}.

Based on this clear precedent, \amici urge this Court to hold that
attachment of a general-purpose computer does not render an abstract idea
patentable, in the present claims or otherwise.
Implementing an
abstract idea in the form of an algorithm on a general-purpose computer is a
``well-understood, routine, conventional activity,'' \clause{mayo at 1300}, that
merely
applies the algorithm in a ``particular technological environment,''
\clause{bilski at 3230 (quoting \sentence{diehr at 191-192})}.
Any ``competent draftsman'' could append elements of a
general-purpose computer to any algorithm.
This case is distinguishable from \inline{diehr}, which found patentable an
algorithm intimately tied to a specialized device, namely a rubber-curing
machine, \clause{see * at 187}, because unlike a rubber-curing machine, a
computer is able to perform any possible algorithm or mathematical
procedure.
Thus, in the claims at issue, the recitation of a general-purpose computer
should not render the claims eligible under \inline{101}.



%%%%%%%%%%%%%%%%%%%%%%%%%%%%%%%%%%%%%%%%%%%%%%%%%%%%%%%%%%%%%%%%%%%%%%%%
%
% SECTION II
%
\section{The Court Should Clarify the Law of Subject Matter Eligibility}

In past decisions, this Court has placed important limits on patentable subject
matter to ensure that the building blocks of invention, including abstract ideas
and laws of nature, remain available to all. \sentence{mayo at 1293}.  The flood
of software patent litigation in recent years, and the ruthless exploitation of
such litigation by patent assertion entities, have made these limitations more
important than ever. \sentence{gao-report at 13}. However, recent Federal
Circuit decisions have weakened these limitations and muddied the waters on
patentable subject matter.  \sentence{gao-report at 13}.

This Court should now reaffirm the importance of \inline{101} limitations and
refute the use of incorrect analytical methods. In particular, this Court should
instruct the lower courts not to rely on three red herrings: details in the
specification, the statutory class of the claims,
or the addition of details of a computer. By advising lower courts to consider
eligibility carefully and thoroughly, this Court will help bring certainty to
patent litigation and relief to innovators. 


%%%%%%%%%%%%%%%%%%%%%%%%%%%%%%%%%%%%%%%%%%%%%%%%%%%%%%%%%%%%%%%%%%%%%%%%
%
% SECTION II.A
%
\subsection{Lower Courts Should Not Rely on
Specification Details to Evaluate Eligibility}

This Court should reaffirm that the proper focus of a \inline{101} subject
matter inquiry is the scope of the patent claims, and that courts should not
rely on language in the specification to decide if claims are sufficiently
concrete. This would resolve a division among the Federal Circuit judges on the
proper role of implementation details in determining eligibility.

It has long been established that ``the claims made in the patent are the sole
measure of the grant.''
\sentence{aro at 339; accord graver-tank at 538-539 (``We have frequently held
that it is the claim which measures the grant to the patentee.''); white at 51
(a claim is not ``like a nose of wax'')}. Since 1836, the Patent Act has
required claims that
``particularly specify and
point out the part, improvement, or combination which he claims as his own
invention or discovery.'' \sentence{keystone at 278;
1870act[S 24] at 201; 112 at /b}. %; see generally sarnoff at 401-402}.
While patent claims are read in light of the
specification and prosecution history, the claims alone determine
patent scope. \sentence{phillips at 1317}.
        
Specifications do not define the scope of an invention; they merely
describe how to make and use the claimed
invention. \sentence{112 at /a}. Thus, specifications, by their nature,
will almost certainly include some detailed, concrete implementations of the
invention at hand.

However, ``the complexity of the
implementing software or the level of detail in the specification does not
transform a claim reciting only an abstract concept into a patent-eligible
system or method.'' \sentence{accenture at 1345; oreilly at 120
(holding ineligible a broad patent claim to electromagnetism, despite a
specification disclosing patent-eligible telegraph technology)}.

It is therefore
critically important that, when deciding whether a patent claim is directed to
patentable subject matter, courts pay close
attention to what is actually being claimed, and be vigilant to ensure that,
regardless of specification details, the claims do not envelop abstract ideas,
laws of nature, or physical phenomena.

Here, Judge Lourie applied the correct approach,
correctly observing that the claims were simply calling for ``a handful of
computer components, in generic, functional terms that would encompass any
device capable of performing the same ubiquitous calculation, storage, and
connectivity functions.'' \sentence{cls-enbanc at 1290}.

Several other judges, however, mistakenly relied on examples in the
specification in determining patent eligibility.
For instance, Judge Rader concluded that the claims
were directed to patentable subject matter, in part by looking to
complex-looking flowcharts and descriptions in the specification. Referring to
one claim, he wrote: ``Lest it be
said that these structural and functional limitations are
mere conventional post-solution activity, \emph{the specification explains
implementation} of the recited special purpose computer system.''
\sentence{cls-enbanc at 1307 (emphasis added)}. The opinion mistakenly relied on
specification details to conclude that the claims are intricate and concrete,
and in so doing missed that the claims themselves still recite basic, abstract
concepts.

This is not the first time the Federal Circuit has made this sort of error.
\Inline{ultramercial} found certain
computer-implemented patent claims to be patent-eligible, and not abstract
ideas, in part because of the ``intricate and complex computer programming''
in the specification. \sentence{* at 1350}. In doing so, the panel missed the
highly abstract nature of the claims, which essentially cover basic e-commerce
concepts. \sentence{see wildtangent-amicus at 8-10}.

If this Court does not reject improper use of the specification
determining subject matter eligibility, the confusion in the case below
and \inline{ultramercial} will allow clever
drafters to circumvent this Court's precedent on abstract ideas,
simply by adding details to the specification. This danger is particularly acute
for software patents, as it is easy to ``recite common language (`boilerplate')
that
describes generic computers'' but that does
not meaningfully limit the claims.
\sentence{browning at 263}. If judges get lost in obfuscatory
language in patent specifications, then many more abstract patents will be
incorrectly held valid, which will sanction improper litigation and
chill innovation.

\Amici urge this Court to instruct the lower courts, in unequivocal
terms, not to rely on specification details to determine subject matter
eligibility.



%%%%%%%%%%%%%%%%%%%%%%%%%%%%%%%%%%%%%%%%%%%%%%%%%%%%%%%%%%%%%%%%%%%%%%%%
%
% SECTION II.B
%
\subsection{System Claims Are Not More Patent-Eligible than Method Claims}
\bblabel[Section \#]{sec2b}

The Court should reiterate that patent subject matter eligibility turns on the
substance and not the form of patent claims, and direct
the lower courts not to decide patent eligibility based on drafting
decisions such as claiming “systems” as opposed to “methods.” 

Despite recent debates regarding the importance of the formal system and method
distinction, it has long been established that drafting formalities should not
distract from
the substantive \inline{101} analysis.  As this Court reaffirmed in
\inline{mayo}, appending to an abstract idea
a phrase such as “apply it” does
not make otherwise ineligible subject matter suddenly patentable. \sentence{*
at 1294}.  To hold differently would permit clever drafting to maneuver around
\inline{101}'s important limitations, and ignore the basic rationale behind this
Court's exceptions to \inline{101}. \sentence{see flook at 590}. One concurring
opinion below, however, urged exactly that wrong approach. \sentence{cls-enbanc
at 1309 (Rader, J.)}.

In articulating these
exceptions, this Court has time and again underscored the principle that patents
must ultimately incentivize innovation. \sentence{mayo at 1293; bilski at 3228
(Kennedy, J., op.)}. While many patents do serve this principle, any claim that
covers an abstract idea, law of nature or physical phenomenon in fact deters
innovation by taking away those “basic tools of scientific and technological
work”  available to all. \sentence{benson at 67}. Any other approach
would create uncertainty and hesitation for innovators, who may decide
that developing technologies or releasing products is prohibitively
expensive. \sentence{see ftcreport at 74}.

One such drafting formality at issue in the case below is the choice of
statutory class, such as a system
claim or method claim.
Such claims are certainly different in theory: the former covers a
machine while the latter covers a process with steps. But, particularly for
software, the difference is merely a drafting exercise, because any method can
be transformed into a system claim by reciting “a computer configured to perform
certain steps” rather than claiming the steps alone. 

Lower courts have long noticed that system and method claims can and sometimes
do identify the same subject matter.  One court, for example, observed
that this Court's precedent on \inline{101} ``applies equally whether an
invention is claimed as an apparatus or process, because the form of the claim
is often an exercise in drafting.'' \sentence{johnson at 1070 (quoted in
\sentence{alappat at 1542})}. Indeed, the Federal Circuit has previously held
that ``the form of the claims should not trump basic issues of patentability.''
\sentence{bancorp at 1277; accord cybersource at 1374 (``[W]e look to the
underlying invention for patent-eligibility purposes.'')}.

Despite this precedent, some judges still rely too heavily on
the statutory class of a claim.  According to Judge Rader,
the system claims of this case “do not claim only an abstract concept without
limitations\ldots because they require a machine.” \sentence{cls-enbanc at
1309}. Thus, Judge Rader held those system claims eligible, while holding
method claims with nearly equivalent wording ineligible. \sentence{see
cls-enbanc at 1312-1313}.

Judge Rader’s reasoning inappropriately turns formalistic drafting practice into
a substantive distinction.  As Judge Lourie observed, the “method and system
claims use similar and often identical language to describe those actions,” so
to have a threshold test turn on this designation would be unreasonable.
\sentence{cls-enbanc at 1289}. He correctly determined that “despite minor
differences in terminology\ldots the asserted method and system claims require
performance of the same basic process.” \sentence{cls-enbanc at 1290}.  To hold
otherwise would allow the form of the claims to trump basic issues of
patentability, and would elevate an exercise in drafting to substantive
significance.

A formalistic approach to system and method claims would be more than a
departure from Supreme Court precedent and long standing tradition; it would
essentially make the abstract ideas exception a dead letter for a huge swath of
patents in the computer arts. Judge Rader’s reasoning implies that a patent
attorney can turn practically any method---abstract or not---into patentable
subject matter simply by relabeling that claim as a “system” with a simple
reference to a computer.  This would contravene the Court's warning
in \inline{flook}
about the danger of the clever draftsman, and it would obliterate the purpose
behind the
abstract ideas exception. This Court should explicitly reject such a result.

\Amici therefore urge this Court to reaffirm that lower courts must examine
system and method claims with equal vigor, and that the entirety of
\inline{101}’s applicability cannot turn on superficial drafting distinctions.  



%%%%%%%%%%%%%%%%%%%%%%%%%%%%%%%%%%%%%%%%%%%%%%%%%%%%%%%%%%%%%%%%%%%%%%%%
%
% SECTION II.C
%
\subsection{Recitation of Details of a General-Purpose Computer Does Not Affect
Eligibility}
\bblabel[Section \#]{sec2c}

This Court should make clear that a clever draftsman cannot turn an abstract
idea into patentable subject matter simply by reciting
aspects of a general-purpose computer, regardless of the level of detail with
which the claims describe the general-purpose computer. Several of the opinions
below were unduly impressed by detailed, technical language that in fact recited
nothing more than parts of a general-purpose computer, \clause{this at
computer-quotes}, and this Court should firmly reject that approach.

Consider a hypothetical ineligible claim to a method of performing long
division using pencil and paper, as \amici discuss above. One could
recite at length the physical attributes of the pencil (``a pencil
comprising a wooden shaft surrounding a cylindrical graphite barrel, the wooden
shaft having a distal end including a rubber eraser, etc.''). But such a
recitation would affect neither the tendency of such a claim to effectively
preempt use of an abstract idea, nor the ineligibility of the claim. Allowing
patent eligibility to turn on this sort of insignificant detail ``would make the
determination of patentable subject matter depend simply on the draftsman's
art,'' a result that this Court should seek to avoid. \sentence{flook at 593}.

Just as details about a pencil should not confer patent
eligibility, neither should details about a general-purpose
computer.
Thus, language from the claims at issue, such as
``data storage unit'' and ``communications controller,'' should not affect the
ineligibility of the claims. This Court should reaffirm this point clearly.

One reason that the lower courts make this error is that they place undue
reliance on the 1994 Federal Circuit decision \inline{alappat}. Although that
case actually dealt with a special form of oscilloscope,
the lower court stated in dicta
that ``a general purpose computer in effect becomes a special purpose computer
once it is programmed'' with software. \sentence{* at 1545}.
Courts have used this statement to support a mistaken conclusion that recitation
of general-purpose computer hardware can confer patent eligibility.
\sentence{see, e.g., cls-enbanc at 1305 (Rader, J.); ultramercial at 1353}.

This reliance on \inline{alappat} is mistaken because the
Court's precedent regarding abstract ideas does not
distinguish between whether something is labeled ``general'' or
``special purpose.'' A
compact disc becomes special-purpose when music is recorded on it, but no patent
should issue on such a ``special-purpose compact disc.'' \sentence{* at
1553-1554 (Archer, J., dissenting)}.

The law of subject matter eligibility does not turn on labels. It
turns on whether a patented claim would preempt virtually all
implementations of an idea, suppressing innovation along the way.
This Court should thus reject the
continued reliance of the lower courts on this dicta from \inline{alappat}.

By ensuring that patent eligibility does not turn on formal drafting practices,
such as recitation of system-style claims or inclusion of details of
general-purpose computer hardware, this Court will take \inline{101} analysis
from the metaphysical confusion that the lower courts have created, and return
it to first principles. At the core of those first principles, which date back
to the drafting of the Constitution, are the imperatives that patents must
be calibrated to promote innovation, and
that the toolbox of abstract ideas must remain available to all. It is these
principles that
should guide the Court's decision.

\clearpage
\part{Conclusion}

For the foregoing reasons, \amici respectfully submit that the Court
should affirm the judgment below.

\signature

\appendix{Implementation of Claim 26 of the '375 Patent in \Numlines Lines of
Computer Code}
\bblabel[Appendix \#]{appx:code}
\appendixpagenumbers

The following \numlines-line computer program, written in the \textsc{basic}
programming language, implements Claim 26 of the '375 Patent.\footnote{A
\textsc{basic} program interpreter to run this program is available at
http://www.vintage-basic.net/.}

\vskip\baselineskip
\inlinebox{\wholeprogram}
\vskip\baselineskip

The text below reviews the elements of the claim in detail and explains how
a general-purpose computer, running the above computer program, would satisfy
all the elements of the claim. For convenience, the entirety of the claim is
reprinted in the next appendix.

All of the computer programming techniques used here predate the patent.
The earliest
possible priority date of the patent is 1992. The \textsc{basic} language dates
back to 1964. \sentence{see basic-manual}. Thus,
the computer techniques used in this brief were
``well-understood, routine, conventional activity
previously engaged in by researchers in the field'' as of the priority date of
the patent. \sentence{mayo at 1294; cf.
cls-enbanc at 1310 (Rader, J.) (asserting that the use of computers in the
claims did not involve such conventional activity)}.

\claimbox{preamble}{\claimelt0}

The preamble recites that the claim covers a general purpose computing system,
called a ``data processing system'' by the claim language. The recitation that
the system is ``to enable the exchange of an obligation'' is a statement of
field of use or intended use, which should not contribute to the scope of the
claim.
\sentence{see bilski at 3231 (``[L]imiting an abstract idea to one field of
use\ldots did not make the concept patentable.''); mpep at S 2103/IC
(instructing that ``statements of intended use or field of use'' ``may raise a
question as to the limiting effect of the language in a claim'')}.

\claimbox{elements 1--2}{\claimelt1\par\claimelt2}

These elements recite general hardware inherent in a general purpose computer. A
``communications controller'' broadly refers to a component of a computer that
receives and processes communications, and a ``first party device'' could refer
to any computer hardware.\footnote{Petitioner has at least once
described the communications controller as a device ``that allows communications
over a wide-area network.'' \sentence{cls-dist95 at 6}. But the text of the
patent belies that limited definition. \sentence{see pat375 at column 7, lines
46-57 (``A number of communications controllers\ldots effect communications
between the processing units and various external hardware devices\ldots. A
large range of communications hardware products are supported, and collectively
are referred to as the stakeholder input/output devices.'' (reference numbers
omitted))}.} A computer must communicate with its users in order to be useful,
so these components are necessary to any computer.

\codeandclaimbox{elements 3--5}{\claimelt3\par\claimelt4\par\claimelt5}{1}{2}

These elements of the claim simply require that a computer store two numbers
representing account balances. The ``data storage unit'' might be any computer
storage component, such as a hard disk or memory. The ``information about'' the
first and third accounts broadly encompasses any account information, such as an
account balance.

The recitations that the information be stored ``independent from'' various
accounts maintained by exchange institutions are statements of intended
use, which should not contribute to the patentability of the claim. Petitioners
have never suggested that the external exchange institutions are necessary
parties to infringement of their claims. Furthermore, so long as the two stored
numbers reflect actual account balances in external banks, the ``independent
from'' limitations are satisfied.

The computer code implements these elements of the claim by instructing a
computer to store two account balances, into variables named \emph{account1} and
\emph{account3}.

\claimbox{element 6}{\claimelt6}

This element is simply further recitation of details about a general purpose
computer. Any computer would necessarily be coupled to a data storage unit, so
that it might access data for processing, and further be coupled to a
communications controller, so that it may receive and output information.

\codeandclaimbox{element 7}{\claimelt7}{3}{3}

According to this element, the computer receives a ``transaction.'' An exchange
of money between two accounts is one type of transaction, and Petitioners have
used described an ``exchange'' as an example of a transaction. (Petr.'s Br.\ 7.)
Thus, this
element requires nothing more than receipt of an instruction to transfer money
between two accounts.

The computer code implements this element by requesting the user to input an
amount of money to
transfer between the first and third account. This is performed by the
\textsc{input} command. Upon running this line of code,
a computer would print out the prompt message, and then await an outside user to
enter a number indicating the amount of money to transfer. The amount to
exchange is stored in a variable named \emph{exchange}.

\codeandclaimbox{element 8}{\claimelt8}{4}{6}

This element describes two operations. First, a computer must check that at
least one of the accounts has a large enough balance to permit the desired
transfer of money (``ensuring that said first party\ldots ha[s] adequate value
in
said first account''). Second, the computer must record the transfer by
adjusting the balances of the accounts (``electronically adjust said first
account and said third account'').

Note the substantial presence of inoperative language in this claim element. The
recitation ``in order to effect an exchange obligation arising from said
transaction between said first party and said second party'' does nothing more
than reiterate that the computer is transferring money between accounts.
Furthermore, the claim recites that the computer must ensure ``adequate value in
said first account and/or said third account,'' and the disjunctive ``and/or''
means that the claim element is satisfied if only one of those accounts is
checked. \sentence{see mpep at S 2103/IC (``Language that suggests or makes
optional but does not require steps to be performed\ldots does not limit the
scope of a claim or claim limitation.'')}.

The computer code implements the step of checking the account balances at line
40, which halts execution (with \textsc{stop}) if the balance of
\emph{account1} is less than the amount to be exchanged. The code implements the
step of effecting the transfer at lines 50--60, which deduct the amount to be
exchanged from \emph{account1} and add that amount to \emph{account3}.

\codeandclaimbox{element 9}{\claimelt9}{7}{7}

This claim element requires only that a computer produce an instruction to
perform the desired transfer of money. The claim element recites ``an
instruction to said first exchange institution and/or said second exchange
institution,'' but the disjunctive ``and/or'' means that a single instruction
suffices. Similarly, the recitation of an instruction ``to adjust said second
account and/or said fourth account'' only requires an instruction with regard to
a single account.

The requirement that the instruction be ``an irrevocable, time invariant
obligation'' is merely a statement of intended use that should not contribute to
the patentability of the claim. An instruction is simply a text, and the
recipient of the instruction chooses whether to treat that text as irrevocable
or time-invariant. Although this claim language could plausibly have been
defined in the specification to require some sort of special format for the
instruction, Petitioners have never identified any such special definition in
any of their briefs to this Court, the Federal Circuit, or the district
court,\footnote{The district court briefs reviewed are identified on the docket
as Documents Nos.\ 53, 54, 68, 95, and 99. The Federal Circuit briefs reviewed
are identified on the docket as Documents Nos.\ 22, 33, 41, and 194.}
and the
specification contains neither term outside of the claims. Furthermore, even if
these terms did have some special meaning, it would only dictate the content of
the instruction text, and content of text does not contribute to patentability.
\sentence{see mpep at S 2106/I (``a mere arrangement of printed matter'' is not
directed to statutory subject matter)}.

The computer code implements this element by causing a computer to print an
instruction to adjust the second account. The instruction directs the
first institution to deduct the amount \emph{exchange} from the account.

\appendix{Claim 26 of the '375 Patent}
\bblabel[Appendix \#]{appx:claim}
\addtotoa{pat375}

\emph{Numbers, in square brackets, have been inserted before each element of the
claim, to assist in referring to claim elements within the brief.}

\wholeclaim

\end{document}

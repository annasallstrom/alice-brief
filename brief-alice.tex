\documentclass{scotus}
\usepackage[ignoremissing]{bluebook}
\usepackage{ifxetex}
\usepackage{xspace}

\usepackage{listings}
%
% This is a really awful hack that causes the listings package to output double
% quotation marks for strings with proper directional quotation marks.
%
\makeatletter
\def\lst@Append#1{%
    \advance\lst@length\@ne
    \if\noexpand#1"%
        \lst@hack@quote
    \else
        \lst@token=\expandafter{\the\lst@token#1}%
    \fi
}
\newif\if@lst@hack@inquote
\def\lst@hack@quote{%
    \if@lst@hack@inquote
        \global\@lst@hack@inquotefalse
        \lst@token=\expandafter{\the\lst@token''}%
    \else
        \global\@lst@hack@inquotetrue
        \lst@token=\expandafter{\the\lst@token``}%
    \fi
}
\makeatother


\defcase{i4i}{
p=Microsoft Corp.,
d=i4i Limited Partnership,
cite=131/S. Ct./2238,
year=2011
}

\defcase{bilski}{
p=Bilski,
d=Kappos,
cite=130 S. Ct. 3218,
year=2010
}

\defstatcode{112p2}{
cite={35/U.S.C./S 112, \P~2},
year=2011
}

\defcase{quanta}{
p={Quanta Computer, Inc.},
d=LG Electronics Corp.,
cite=553/U.S./617,
year=2008
}

\defstatcode{101}{
cite=35 U.S.C. S 101,
year=2013
}

\defcase{flook}{
p=Parker,
d=Flook,
cite=437 U.S. 584,
year=1978,
inlinedefendant
}

\defcase{funkbros}{
p=Funk Brothers Seed Co.,
d=Kalo Inoculant Co.,
cite=333 U.S. 127,
year=1948
}

\defcase{leroy}{
p=Le Roy,
d=Tatham,
year=1852,
cite=55 U.S. (14 How.) 156
}

\defcase{benson}{
p=Gottschalk,
d=Benson,
inlinedefendant,
cite=409 U.S. 63,
year=1972
}

\defcase{mayo}{
p=Mayo Collaborative Services,
d={Prometheus Laboratories, Inc.},
cite=132 S. Ct. 1289,
inline=Mayo,
year=2012
}

\defcase{cls-enbanc}{
p=CLS Bank International,
d=Alice Corp.,
cite=717 F.3d 1269,
year=2013,
court=Fed. Cir.,
enbanc
}

\defpatent{pat375}{
number={7725375},
date={June 27, 2005}
}

\ExplanatoryPhrase{vacated and rehearing en banc granted,}

\defcase{cls-panel}{
p=CLS Bank International,
d=Alice Corp.,
cite=685 F.3d 1341,
year=2012,
court=Fed. Cir.,
paren=panel decision,
subsequent={vacated and rehearing en banc granted, cls-vacate}
}
\defcase{cls-vacate}{
sameparties=cls-panel,
cite=484 Fed. Appx. 559,
year=2012,
court=Fed. Cir.
}

\defcase{diehr}{
p=Diamond,
d=Diehr,
cite=450 U.S. 175,
inlinedefendant,
year=1981
}
\defbook{mpep}{
    title=Manual of Patent Examining Procedure,
    instauth=United States Patent and Trademark Office,
    year=2012,
    edition={8th ed., 9th rev.},
    hereinafter=MPEP
}

\defcasedoc{cls-dist95}{
sameparties=cls-panel,
docket=1:07-cv-974,
cite=768 F. Supp. 2d 221,
docname=Alice Corp. Pty. Ltd.'s Renewed Cross-Motion for Partial Summary
Judgment as to Subject Matter Eligibility,
date={Sept. 22, 2010},
court=Dist. D.C.,
paren=Doc.\ No.\ 95
}
\defcasedoc{cls-dist53}{
sameparties=cls-panel,
docket=1:07-cv-974,
cite=768 F. Supp. 2d 221,
court=Dist. D.C.,
docname=Alice Corp. Pty. Ltd.'s Cross-Motion for Partial Summary
Judgment as to Subject Matter Eligibility,
date={Apr. 3, 2009},
paren=Doc.\ No.\ 53
}
\defcasedoc{cls-dist54}{
sameparties=cls-panel,
docket=1:07-cv-974,
cite=768 F. Supp. 2d 221,
docname=Alice Corp. Pty. Ltd.'s Cross-Motion for Partial Summary
Judgment as to Subject Matter Eligibility,
date={Apr. 3, 2009},
court=Dist. D.C.,
paren=Doc.\ No.\ 54
}
\defcasedoc{cls-dist68}{
sameparties=cls-panel,
docket=1:07-cv-974,
cite=768 F. Supp. 2d 221,
court=Dist. D.C.,
docname=Alice Corp. Pty. Ltd.'s Reply Memorandum in Support of its Cross-Motion
for Partial Summary Judgment as to Patent-Eligibility,
date={May 19, 2009},
paren=Doc.\ No.\ 68
}
\defcasedoc{cls-dist99}{
sameparties=cls-panel,
docket=1:07-cv-974,
cite=768 F. Supp. 2d 221,
court=Dist. D.C.,
docname=Alice Corp. Pty. Ltd.'s Reply Memorandum in Support of its Renewed
Cross-Motion for Partial Summary Judgment as to Patent-Eligibility,
date={Oct. 22, 2010},
paren=Doc.\ No.\ 99
}
\defcasedoc{cls-app22}{%
sameparties=cls-panel,
docket=11-1301,
cite=685 F.3d 1341,
court=Fed. Cir.,
docname=Brief of Defendant-Appellant Alice Corp. Pty. Ltd.,
date={June 24, 2011},
paren=Doc.\ No.\ 22
}
\defcasedoc{cls-app33}{%
sameparties=cls-panel,
docket=11-1301,
cite=685 F.3d 1341,
court=Fed. Cir.,
docname=Reply Brief of Defendant-Appellant Alice Corp. Pty. Ltd.,
date={Sept. 16, 2011},
paren=Doc.\ No.\ 33
}
\defcasedoc{cls-app194}{
sameparties=cls-panel,
docket=11-1301,
cite=717 F.3d 1269,
court=Fed. Cir.,
docname=\emph{En Banc} Response Brief of Defendant-Appellant Alice Corp. Pty.
Ltd.,
date={Jan. 10, 2013},
paren=Doc.\ No.\ 194
}

\newcounter{claimcounter}
\def\theclaim{%
A data processing system to enable the exchange of an obligation
between parties, the system comprising:
  \element a communications controller,
  \element a first party device, coupled to said communications controller,
  \element a data storage unit having stored therein
    \element\sub (a)\ind information about a first account for a first party,
    independent from a second account maintained by a first exchange
    institution, and
    \element\sub (b)\ind information about a third account for a second party,
    independent from a fourth account maintained by a second exchange
    institution; and
  \element a computer, coupled to said data storage unit and said
  communications controller, that is configured to
    \element\sub (a)\ind receive a transaction from said first party device via
    said communications controller;
    \element\sub (b)\ind electronically adjust said first account and said
    third account in order to effect an exchange obligation arising from said
    transaction between said first party and said second party after ensuring
    that said first party and/or said second party have adequate value in said
    first account and/or said third account, respectively; and
    \element\sub (c)\ind generate an instruction to said first exchange
    institution and/or said second exchange institution to adjust said second
    account and/or said fourth account in accordance with the adjustment of said
    first account and/or said third account, wherein said instruction being an
    irrevocable, time invariant obligation placed on said first exchange
    institution and/or said second exchange institution.
}
\newcount\claimeltcount
\def\claimelt#1{%
    \begingroup
        \let\element\or
        \let\sub\relax
        \let\preamble\relax
        \let\ind\space
        \claimeltcount=#1\relax
        \expandafter\ifcase\expandafter\claimeltcount\theclaim\fi
        \unskip
    \endgroup
}
\def\wholeclaim{%
    \begingroup
        \let\preamble\par
        \def\element{\par\endgroup\begingroup\advance\leftskip\parindent}%
        \def\sub{\advance\leftskip\parindent}%
        \let\ind\quad
        \begingroup\theclaim\par\endgroup
    \endgroup
}


%
% If you do not have these fonts, comment the next few lines out by changing
% \ifxetex to \iffalse
%
\ifxetex
    \usepackage{fontspec}
    \setmainfont[BoldFont=ITC Century Std-Bold,Ligatures=TeX]{%
        Century Expanded LT Std%
    }
    \def\titlefont{\fontspec{Cloister Black}\LARGE}
    \def\basickeywordstyle{\addfontfeatures{Letters=UppercaseSmallCaps}}
\else
    \def\basickeywordstyle{\bfseries}
\fi

%
% Macros for how long the computer program is
%
\def\numlines{seven\xspace}
\def\Numlines{Seven\xspace}
\def\wholeprogram{\programlines{1}{7}}

%
% Macros to format the claim language and/or computer program
%
\def\programlines#1#2{%
    \lstset{keywordstyle=\basickeywordstyle,identifierstyle=\itshape}
    \lstset{breaklines=true,showstringspaces=false}
    \lstset{aboveskip=0pt,belowskip=0pt}
    \lstset{language=[Visual]Basic,columns=fullflexible}
    \lstinputlisting[firstline=#1,lastline=#2]{%
        implementations/375claim26.bas%
    }%
}
\long\def\inlinebox#1{%
    \vskip\parskip
    \hrule\kern1.5pt \hrule\nobreak
    \begingroup
        \vskip 4pt
        \parskip=0pt
        #1\par
        \nobreak\vskip 4pt
    \endgroup
    \hrule\kern1.5pt \hrule
}
\long\def\claimbox#1#2{%
    \vskip.5\baselineskip
    \inlinebox{%
        \noindent\textsc{Claim 26, #1:}\par
        {\itshape#2\unskip}%
    }%
}
\long\def\codeandclaimbox#1#2#3#4{%
    \vskip.5\baselineskip
    \inlinebox{%
        \noindent\textsc{Claim 26, #1:}\par
        {\itshape#2\unskip}%
        \vskip 4pt
        \hrule
        \vskip 4pt
        \noindent
        \ifnum#3=#4\relax
            \textsc{Computer code, line #30:}\par
        \else
            \textsc{Computer code, lines #30--#40:}\par
        \fi
        \programlines{#3}{#4}%
    }%
}%


\docket{13-298}
\casecaption{
\textsc{Alice Corporation Pty. Ltd.},\\\The{Petitioner,}\\v.\\
\textsc{CLS Bank International and CLS Services Ltd.},\\
\The{Respondent.}
}

\posture{On Writ of Certiorari \\
to the United States Court of Appeals \\
for the Federal Circuit}

\counsel{
\textsc{Charles Duan} \\
\counselofrecord \\
\textsc{Public Knowledge} \\
1818 N St NW, Suite 410 \\
Washington, DC 20036 \\
(202) 861-0020 \\
cduan@publicknowledge.org \\
\whois{Counsel for Amicus Curiae}
}

\title{Brief of Public Knowledge as
\protect\\\emph{Amicus Curiae} in Support of Respondent}

\begin{document}

\maketitle

\romanpagenumbers
\tableofcontents

\part{Table of Authorities}

\tableofauthorities

\clearpage
\arabicpagenumbers

\part{Interest of \emph{Amicus Curiae}}

Public Knowledge is a non-profit organization that is dedicated to preserving
the openness of the Internet and the public's access to knowledge; promoting
creativity through balanced intellectual property rights; and upholding and
protecting the rights of consumers to use innovative technology lawfully. As
part of this mission, Public Knowledge advocates on behalf of the public
interest for a balanced patent system, particularly with respect to new and
emerging technologies.

Public Knowledge has previously served as \emph{amicus} in key patent cases.
\sentence{e.g., i4i; bilski; quanta}.

\fullcite{i4i}
\fullcite{bilski}
\fullcite{quanta}

\part{Summary of Argument}

*

\part{Argument}

\section{The Claims at Issue Are Ineligible Under Section 101 Because They
Effectively Preempt Substantially All Uses of an Abstract Idea}

\subsection{A \Numlines-Line Computer Implementation of the Patented Technology
Illustrates that the Claims Are Not Meaningfully Limited Beyond an Abstract
Idea}

The claims use complex, technical-sounding language like ``shadow accounts''
that make the claim to appear substantially limited beyond a mere abstract idea.
However, a careful reading of the claims shows that this complex language does
not in fact actually provide such substantial limitations.

To demonstrate this, we prepare a computer program that implements all the
features of the claims. The computer program is very short, indicating that the
verbose language of the claims does not in fact demand specific, particular
implementations but rather can expansively cover all implementations.

\begingroup
\lstset{keywordstyle=\basickeywordstyle,identifierstyle=\itshape}
\lstset{breaklines=true,showstringspaces=false}
\lstset{language=[Visual]Basic,columns=fullflexible}
\lstinputlisting[lastline=7]{implementations/375claim26.bas}
\endgroup

Certain judges of the Federal Circuit were clearly misled by the claim language.
They believed that the claim required particular, specific implementation
details, due to the apparently technical language of the claims and the patent
specification. However, our presented computer implementation shows these
beliefs to be in error.

\subsection{Read with Proper Expansiveness, the Claims Cover Substantially All
Computer Implementations of a Basic, Abstract Accounting Idea of Third-Party
Escrow}

As can be plainly seen, the computer program presented is nothing more than the
basic steps of accounting performed by a third-party escrow. All of the steps
relate either directly to inherent aspects of this idea or insignificant pre- or
post-solution activity.

Since the computer program implements all the features of the claims at issue,
this necessarily means that the patent claims are directed to nothing more than
the abstract idea of accounting by a third-party escrow. Accordingly,
practically anyone implementing this abstract idea would need to include the
same steps as those in our computer program, and thus would infringe the claims
of the patent.

Thus, the patent claims preempt essentially all computer-implemented uses of
accounting by a third-party escrow.

\subsection{The Court Should Disapprove the Preemption of Substantially All
Computer Uses of an Abstract Idea, and Thus Hold the Claims at Issue Ineligible}

Under this Court's precedent, a patent claim is ineligible under \inline{101} if
that claim has the practical effect of removing all uses of an abstract idea
from the public domain. In the present case, the claim would have the practical
effect of removing all uses of an abstract idea \emph{on a general purpose
computer} from the public domain. The Court should find such a claim ineligible
as well, for the following reasons.

First, the addition of a general purpose computer is no more significant than
the addition of post-solution or pre-solution activity that the Court has
previously held not to render an otherwise abstract idea patentable. Cases such
as \inline{mayo} and \inline{flook} have disregarded post-solution and
pre-solution activity on the rationale that
%
% Note: I have no idea if this is actually the rationale
[it does not significantly contribute to the subject matter of the claim, and
that any competent draftsman could render a claim patentable by simple and
insignificant addition of such limitations].
The same applies to the addition of a general-purpose computer. It does not
contribute significantly to the inventive aspect of the claim, and any competent
draftsman could easily insert a general purpose computer.

Furthermore, the Court has reasoned that abstract ideas must remain unpatentable
to ensure that the basic tools of innovation remain free to all. \sentence{see
bilski; mayo; benson}. Allowing patents on those basic tools of innovation would
hinder, rather than promote, the progress of technology. Computers are also a
basic tool of innovation, which enable software developers to test out new
ideas, improve on existing ones, and create new innovations. They are essential
to the progress of technology.  Allowing patents on abstract ideas merely
implemented on general purpose computers would thus equally hinder the progress
of technology.

Finally, strong policy considerations indicate that mere inclusion of a general
purpose computer should not render an otherwise abstract idea patentable.
Computers are in widespread use today, and are effectively unavoidable. Thus,
while as a truly formal matter a general purpose computer is only one possible
way of implementing an abstract idea, in a practical sense a general purpose
computer is the only way of implementing almost any of the abstract ideas used
in society today. Condemning the public to resort to pencil and paper to avoid
patent infringement is a simply untenable demand.

%
% This argument needs further elaboration.
%
The Federal Circuit's holding in \inline{alappat} should not affect this
conclusion. There, the lower court held that a general-purpose computer, when
instructed to perform a specific program, becomes a special-purpose computer.
This is simply a statement of how a computer works internally, and says nothing
about the degree to which a patent claim is limited by incorporation of a
general purpose computer.

%
% This may better belong in II.C.
%
For similar reasons, the mere recitation in a patent claim of computer hardware,
not specifically related to the inventive aspects of the claim, should not
affect eligibility of the claim. This would enable a clever draftsman to evade
\inline{101}.

\section{The Court Should Proactively Clarify the Law of Subject Matter
Eligibility in Order to Avoid Further Errors Relating to Abstract Ideas}

The Supreme Court has taken numerous subject matter eligibility cases recently.
It does so because the Federal Circuit is in a confused state about the law of
\inline{101}, primarily because a small faction of that court repeatedly applies
incorrect analytical techniques to improperly find patents eligible even when
this Court's precedents demand otherwise.

To clearly enunciate the law for the Federal Circuit and to prevent the need for
further appeals, this Court should explicitly reject those improper analytical
techniques, some of which have been catalogued below.

\subsection{The Court Should Enunciate the Inappropriateness of Using
Specification Details to Evaluate Subject Matter Eligibility}

In assessing whether a claim is ineligible under \inline{101}, courts must
consider the entire breadth of the claim. Claims directed to an abstract idea
will still cover specific, concrete implementations of that abstract idea, so
the mere fact that a claim covers a concrete implementation is no indicator that
a claim is directed to eligible subject matter.

Nevertheless, certain judges of the Federal Circuit persistently err by relying
on specific examples to find patent claims eligible. In the present case, the
plurality opinion justified its finding that the system claims of the patents at
issue were eligible, by selecting a complex-looking flowchart from the
specification to point to the supposed complexity and concreteness of the claim.
By doing so, they failed to contemplate the possibility that other, simpler,
abstract ideas were \emph{also} covered by that same claim---ideas such as the
14-line computer program presented in this brief.

Ironically, those same judges of the Federal Circuit criticize their opposed
colleagues for failing to read the ``claims as a whole.'' It is in fact those
opposed colleagues who have actually read the claims as a whole, contemplating
the vast scope of what they cover. It is that plurality of the Federal Circuit,
instead, who fails to read the claims as a whole, focusing wrongly on specific
examples and obfuscatory language that misleadingly make abstract ideas appear
patentable.

\subsection{The Court Should Reaffirm its Longstanding View that Mere Drafting
Decisions, such as Choosing Between System and Method Claims, Do Not Affect
Subject Matter Eligibility}

The formalistic approach favored by some judges of the Federal Circuit lends to
easy circumvention by clever patent drafting. For example, the suggestion that
the method claims in the present case are ineligibile, while system claims
directed to the same technology are eligible, simply encourages patent
applicants to use system claims in order to skirt the abstract ideas test.

Granting such weight to mere formal drafting practices ignores the basic
rationale behind the Supreme Court's exceptions to \inline{101}. In explaining
the basis for the three exceptions to \inline{101}, this Court has applied the
fundamental principle that patents must ultimately incentivize innovation. While
patents on many inventions do serve this principle, patents to abstract ideas,
laws of nature and physical phenomena would in fact deter innovation by taking
away those ``basic tools of research available to all.''

Several judges of the Federal Circuit ignore this basic rationale. Judge Rader,
for example, has intimated that the three exceptions to \inline{101} are
essentially tautological, because one ``cannot invent an abstract idea, law of
nature or physical phenomenon'' since they have been around the whole time.

This unduly narrow, formalistic view of the exceptions to \S~101 fails to
adequately protect the concerns about incentives for innovation explicitly
relied upon by this Court. Under Judge Rader's view, mere addition of even the
most insignificant step to an otherwise abstract method would suddenly make that
abstract method patentable, because the combination would not have existed
before. The Court has specifically denounced this possibility, in holding
numerous times that insignificant post-solution activity and pre-solution
activity cannot render an otherwise abstract idea patentable.

\subsection{Recitation of Basic, Widely Available Platform Technologies,
Regardless of Detail, Cannot Render an Abstract Idea Patentable}

The Federal Circuit repeatedly cites recitations of basic general purpose
computing hardware as evidence that a claim is directed to eligible
subject matter under \inline{101}. This is often done by overstating this
court's dicta in \inline{bilski}, that the ``machine or transformation'' test is
an ``important clue'' in assessing subject matter eligibility.

The Court should clarify that mere recitation of general purpose platform
technologies, such as general purpose computers, cannot render an otherwise
ineligible claim eligible. Such a holding would be consistent with this Court's
precedent, and more importantly would strongly advance the principles of
incentivizing innovation, by protecting those ``basic tools of innovation''
meant to be ``available to all.''

As an analogy, consider a claim directed to the basic idea of addition,
performed with paper and pencil. The paper and pencil could be described in
great detail:
\begin{quote}
Drawing one or more numerical figures, with a pencil comprising a wooden shaft
substantially in the shape of a hexagonal prism, the wooden shaft surrounding a
cylindrical graphite barrel, the wooden shaft having a distal end including a
rubber eraser, the wooden shaft further having a proximal end sharpened to
thereby expose a portion of the cylindrical graphite barrel.
\end{quote}
Such a claim would certainly satisfy the machine-or-transformation test (a
pencil is a machine of sorts, and the adherence of graphite to paper would
constitute transformation of matter, among other things), but certainly such a
claim would not be eligible subject matter, regardless of the level of detail.
This is because paper and pencil are the basic tools of invention. To permit the
patenting of abstract ideas merely tied to such basic tools would be tantamount
to permitting the patenting of those abstract ideas alone.

Certain judges of the Federal Circuit criticize this approach, believing that it
improperly imports questions of novelty and obviousness into \inline{101}.
However, as this Court's precedent makes clear, this is not the case.
\sentence{see flook}.

\part{Conclusion}

For the foregoing reasons, \emph{amicus} respectfully submits that the Court
should affirm the district court.

\signature

\appendix{Implementation of Claim 26 of the '375 Patent in \Numlines Lines of
Computer Code}

The following \numlines-line computer program, written in the \textsc{basic}
programming language, implements Claim 26 of the '375 Patent.

\inlinebox{\wholeprogram}

The subsequent text reviews the elements of the claim in detail and explains how
a general-purpose computer, running the above computer program, would satisfy
all the elements of the claim. For convenience, the entirety of the claim is
reprinted in the next appendix.

\claimbox{preamble}{\claimelt0}

The preamble recites that the claim covers a general purpose computing system,
called a ``data processing system'' by the claim language. The recitation that
the system is ``to enable the exchange of an obligation'' is a statement of
intended use, which should not contribute to the scope of the claim.

\claimbox{elements 1--2}{\claimelt1\par\claimelt2}

These elements recite general hardware inherent in a general purpose computer. A
``communications controller'' broadly refers to a component of a computer that
receives and processes communications, and a ``first party device'' could refer
to any computer hardware. A standard keyboard could potentially satisfy this
limitation.

\codeandclaimbox{elements 3--5}{\claimelt3\par\claimelt4\par\claimelt5}{1}{2}

These elements of the claim simply require that a computer store two numbers
representing account balances. The ``data storage unit'' might be any computer
storage component, such as a hard disk or memory. The ``information about'' the
first and third accounts broadly encompass any account information, such as an
account balance.

The recitations that the information be stored ``independent from'' various
accounts maintained by exchange institutions are simply statements of intended
use, which should not contribute to the patentability of the claim. Petitioners
have never suggested that the external exchange institutions are necessary
parties to infringement of their claims.

The computer code implements these elements of the claim by instructing a
computer to store two account balances, into variables named \emph{account1} and
\emph{account3}.

\claimbox{element 6}{\claimelt6}

This element is simply further recitation of details about a general purpose
computer. Any computer would necessarily be coupled to a data storage unit, so
that it might access data for processing, and further be coupled to a
communications controller, so that it may receive and output information.

\codeandclaimbox{element 7}{\claimelt7}{3}{3}

According to this element, the computer receives a ``transaction.'' An exchange
of money between two accounts is one type of transaction. Thus, this element
requires nothing more than receipt of an instruction to transfer money between
two accounts.

The computer code implements this by requesting input of an amount of money to
transfer between the first and third account. Upon running this line of code,
a computer would print out a prompt message, and then await an outside user to
enter a number indicating the amount of money to transfer. The amount to
exchange is stored in a variable named \emph{exchange}.

\codeandclaimbox{element 8}{\claimelt8}{4}{6}

This element describes two operations. First, a computer must check that at
least one of the accounts has a large enough balance to permit the desired
transfer of money (``ensuring that said first party...ha[s] adequate value in
said first account''). Second, the computer must record the transfer by
adjusting the balances of the accounts (``electronically adjust said first
account and said third account'').

Note the substantial presence of inoperative language in this claim element. The
recitation ``in order to effect an exchange obligation arising from said
transaction between said first party and said second party'' does nothing more
than reiterate that the computer is transferring money between accounts.
Furthermore, the claim recites that the computer must ensure ``adequate value in
said first account and/or said third account,'' and the disjunctive ``and/or''
means that the claim element is satisfied if only one of those accounts is
checked.

The computer code implements the step of checking the account balances at line
40, which halts execution (with \textsc{stop}) if the balance of
\emph{account1} is less than the amount to be exchanged. The code implements the
step of effecting the transfer at lines 50--60, which deducts the amount to be
exchanged from \emph{account1} and adds that amount to \emph{account3}.

\codeandclaimbox{element 9}{\claimelt9}{7}{7}

This claim element requires only that a computer output an instruction to
perform the desired transfer of money. The claim element recites ``an
instruction to said first exchange institution and/or said second exchange
institution,'' but the disjunctive ``and/or'' means that a single instruction
suffices. Similarly, the recitation of an instruction ``to adjust said second
account and/or said fourth account'' only requires an instruction with regard to
a single account.

The requirement that the instruction be ``an irrevocable, time invariant
obligation'' is merely a statement of intended use that should not contribute to
the patentability of the claim. An instruction is simply a text, and the
recipient of the instruction chooses whether to treat that text as irrevocable
or time-invariant. Although this claim language could plausibly have been
defined in the specification to require some sort of special format for the
instruction, Petitioners have never identified any such special definition in
any of their briefs to this Court or the Federal Circuit, and the text of the
specification contains neither term outside of the claims. Furthermore, even if
these terms did have some special meaning, it would only dictate the content of
the instruction text, and content of text does not contribute to patentability.

The computer code implements this element by causing a computer to print an
instruction to adjust the second account. The instruction directs the
first institution to deduct the amount \emph{exchange} from the account.

\appendix{Claim 26 of the '375 Patent}

\wholeclaim

\end{document}

\documentclass{scotus}
\usepackage[ignoremissing]{bluebook}
\usepackage{ifxetex}
\usepackage{xspace}

\usepackage{listings}
%
% This is a really awful hack that causes the listings package to output double
% quotation marks for strings with proper directional quotation marks.
%
\makeatletter
\def\lst@Append#1{%
    \advance\lst@length\@ne
    \if\noexpand#1"%
        \lst@hack@quote
    \else
        \lst@token=\expandafter{\the\lst@token#1}%
    \fi
}
\newif\if@lst@hack@inquote
\def\lst@hack@quote{%
    \if@lst@hack@inquote
        \global\@lst@hack@inquotefalse
        \lst@token=\expandafter{\the\lst@token''}%
    \else
        \global\@lst@hack@inquotetrue
        \lst@token=\expandafter{\the\lst@token``}%
    \fi
}
\makeatother


\defcase{i4i}{
p=Microsoft Corp.,
d=i4i Limited Partnership,
cite=131/S. Ct./2238,
year=2011
}

\defcase{bilski}{
p=Bilski,
d=Kappos,
cite=130 S. Ct. 3218,
year=2010
}

\defstatcode{112p2}{
cite={35/U.S.C./S 112, \P~2},
year=2011
}

\defcase{quanta}{
p={Quanta Computer, Inc.},
d=LG Electronics Corp.,
cite=553/U.S./617,
year=2008
}

\defstatcode{101}{
cite=35 U.S.C. S 101,
year=2013
}

\defcase{flook}{
p=Parker,
d=Flook,
cite=437 U.S. 584,
year=1978,
inlinedefendant
}

\defcase{funkbros}{
p=Funk Brothers Seed Co.,
d=Kalo Inoculant Co.,
cite=333 U.S. 127,
year=1948
}

\defcase{leroy}{
p=Le Roy,
d=Tatham,
year=1852,
cite=55 U.S. (14 How.) 156
}

\defcase{benson}{
p=Gottschalk,
d=Benson,
inlinedefendant,
cite=409 U.S. 63,
year=1972
}

\defcase{mayo}{
p=Mayo Collaborative Services,
d={Prometheus Laboratories, Inc.},
cite=132 S. Ct. 1289,
inline=Mayo,
year=2012
}

\defcase{cls-enbanc}{
p=CLS Bank International,
d=Alice Corp.,
cite=717 F.3d 1269,
year=2013,
court=Fed. Cir.,
enbanc
}

\defpatent{pat375}{
number={7725375},
date={June 27, 2005}
}

\ExplanatoryPhrase{vacated and rehearing en banc granted,}

\defcase{cls-panel}{
p=CLS Bank International,
d=Alice Corp.,
cite=685 F.3d 1341,
year=2012,
court=Fed. Cir.,
paren=panel decision,
subsequent={vacated and rehearing en banc granted, cls-vacate}
}
\defcase{cls-vacate}{
sameparties=cls-panel,
cite=484 Fed. Appx. 559,
year=2012,
court=Fed. Cir.
}

\defcase{diehr}{
p=Diamond,
d=Diehr,
cite=450 U.S. 175,
inlinedefendant,
year=1981
}
\defbook{mpep}{
    title=Manual of Patent Examining Procedure,
    instauth=United States Patent and Trademark Office,
    year=2012,
    edition={8th ed., 9th rev.},
    hereinafter=MPEP
}

\defcasedoc{cls-dist95}{
sameparties=cls-panel,
docket=1:07-cv-974,
cite=768 F. Supp. 2d 221,
docname=Alice Corp. Pty. Ltd.'s Renewed Cross-Motion for Partial Summary
Judgment as to Subject Matter Eligibility,
date={Sept. 22, 2010},
court=Dist. D.C.,
paren=Doc.\ No.\ 95
}
\defcasedoc{cls-dist53}{
sameparties=cls-panel,
docket=1:07-cv-974,
cite=768 F. Supp. 2d 221,
court=Dist. D.C.,
docname=Alice Corp. Pty. Ltd.'s Cross-Motion for Partial Summary
Judgment as to Subject Matter Eligibility,
date={Apr. 3, 2009},
paren=Doc.\ No.\ 53
}
\defcasedoc{cls-dist54}{
sameparties=cls-panel,
docket=1:07-cv-974,
cite=768 F. Supp. 2d 221,
docname=Alice Corp. Pty. Ltd.'s Cross-Motion for Partial Summary
Judgment as to Subject Matter Eligibility,
date={Apr. 3, 2009},
court=Dist. D.C.,
paren=Doc.\ No.\ 54
}
\defcasedoc{cls-dist68}{
sameparties=cls-panel,
docket=1:07-cv-974,
cite=768 F. Supp. 2d 221,
court=Dist. D.C.,
docname=Alice Corp. Pty. Ltd.'s Reply Memorandum in Support of its Cross-Motion
for Partial Summary Judgment as to Patent-Eligibility,
date={May 19, 2009},
paren=Doc.\ No.\ 68
}
\defcasedoc{cls-dist99}{
sameparties=cls-panel,
docket=1:07-cv-974,
cite=768 F. Supp. 2d 221,
court=Dist. D.C.,
docname=Alice Corp. Pty. Ltd.'s Reply Memorandum in Support of its Renewed
Cross-Motion for Partial Summary Judgment as to Patent-Eligibility,
date={Oct. 22, 2010},
paren=Doc.\ No.\ 99
}
\defcasedoc{cls-app22}{%
sameparties=cls-panel,
docket=11-1301,
cite=685 F.3d 1341,
court=Fed. Cir.,
docname=Brief of Defendant-Appellant Alice Corp. Pty. Ltd.,
date={June 24, 2011},
paren=Doc.\ No.\ 22
}
\defcasedoc{cls-app33}{%
sameparties=cls-panel,
docket=11-1301,
cite=685 F.3d 1341,
court=Fed. Cir.,
docname=Reply Brief of Defendant-Appellant Alice Corp. Pty. Ltd.,
date={Sept. 16, 2011},
paren=Doc.\ No.\ 33
}
\defcasedoc{cls-app194}{
sameparties=cls-panel,
docket=11-1301,
cite=717 F.3d 1269,
court=Fed. Cir.,
docname=\emph{En Banc} Response Brief of Defendant-Appellant Alice Corp. Pty.
Ltd.,
date={Jan. 10, 2013},
paren=Doc.\ No.\ 194
}

\newcounter{claimcounter}
\def\theclaim{%
A data processing system to enable the exchange of an obligation
between parties, the system comprising:
  \element a communications controller,
  \element a first party device, coupled to said communications controller,
  \element a data storage unit having stored therein
    \element\sub (a)\ind information about a first account for a first party,
    independent from a second account maintained by a first exchange
    institution, and
    \element\sub (b)\ind information about a third account for a second party,
    independent from a fourth account maintained by a second exchange
    institution; and
  \element a computer, coupled to said data storage unit and said
  communications controller, that is configured to
    \element\sub (a)\ind receive a transaction from said first party device via
    said communications controller;
    \element\sub (b)\ind electronically adjust said first account and said
    third account in order to effect an exchange obligation arising from said
    transaction between said first party and said second party after ensuring
    that said first party and/or said second party have adequate value in said
    first account and/or said third account, respectively; and
    \element\sub (c)\ind generate an instruction to said first exchange
    institution and/or said second exchange institution to adjust said second
    account and/or said fourth account in accordance with the adjustment of said
    first account and/or said third account, wherein said instruction being an
    irrevocable, time invariant obligation placed on said first exchange
    institution and/or said second exchange institution.
}
\newcount\claimeltcount
\def\claimelt#1{%
    \begingroup
        \let\element\or
        \let\sub\relax
        \let\preamble\relax
        \let\ind\space
        \claimeltcount=#1\relax
        \expandafter\ifcase\expandafter\claimeltcount\theclaim\fi
        \unskip
    \endgroup
}
\def\wholeclaim{%
    \begingroup
        \let\preamble\par
        \def\element{\par\endgroup\begingroup\advance\leftskip\parindent}%
        \def\sub{\advance\leftskip\parindent}%
        \let\ind\quad
        \begingroup\theclaim\par\endgroup
    \endgroup
}


%
% If you do not have these fonts, comment the next few lines out by changing
% \ifxetex to \iffalse
%
\ifxetex
    \usepackage{fontspec}
    \setmainfont[BoldFont=ITC Century Std-Bold,Ligatures=TeX]{%
        Century Expanded LT Std%
    }
    \def\titlefont{\fontspec{Cloister Black}\LARGE}
    \def\basickeywordstyle{\addfontfeatures{Letters=UppercaseSmallCaps}}
\else
    \def\basickeywordstyle{\bfseries}
\fi

%
% Macros for how long the computer program is
%
\def\numlines{seven\xspace}
\def\Numlines{Seven\xspace}
\def\wholeprogram{\programlines{1}{7}}

%
% Macros to format the claim language and/or computer program
%
\def\programlines#1#2{%
    \lstset{keywordstyle=\basickeywordstyle,identifierstyle=\itshape}
    \lstset{breaklines=true,showstringspaces=false}
    \lstset{aboveskip=0pt,belowskip=0pt}
    \lstset{language=[Visual]Basic,columns=fullflexible}
    \lstinputlisting[firstline=#1,lastline=#2]{%
        implementations/375claim26.bas%
    }%
}
\long\def\inlinebox#1{%
    \vskip\parskip
    \hrule\kern1.5pt \hrule\nobreak
    \begingroup
        \vskip 4pt
        \parskip=0pt
        #1\par
        \nobreak\vskip 4pt
    \endgroup
    \hrule\kern1.5pt \hrule
}
\long\def\claimtext#1#2{%
    \noindent
    \bblabel{appx:claim:#1}%
    \textsc{Claim 26, #1:}\par
    {\itshape#2\unskip}%
}
\long\def\claimbox#1#2{%
    \vskip.5\baselineskip
    \inlinebox{%
        \claimtext{#1}{#2}%
    }%
}
\long\def\codeandclaimbox#1#2#3#4{%
    \vskip.5\baselineskip
    \inlinebox{%
        \claimtext{#1}{#2}%
        \vskip 4pt
        \hrule
        \vskip 4pt
        \noindent
        \ifnum#3=#4\relax
            \textsc{Computer code, line #30:}\par
        \else
            \textsc{Computer code, lines #30--#40:}\par
        \fi
        \programlines{#3}{#4}%
    }%
}%

%
% Other useful macros
%
\def\Amicus{\emph{Amicus}\xspace}
\def\amicus{\emph{amicus}\xspace}

\docket{13-298}
\casecaption{
\textsc{Alice Corporation Pty. Ltd.},\\\The{Petitioner,}\\v.\\
\textsc{CLS Bank International and CLS Services Ltd.},\\
\The{Respondent.}
}

\posture{On Writ of Certiorari \\
to the United States Court of Appeals \\
for the Federal Circuit}

\counsel{
\textsc{Charles Duan} \\
\counselofrecord \\
\textsc{Public Knowledge} \\
1818 N St NW, Suite 410 \\
Washington, DC 20036 \\
(202) 861-0020 \\
cduan@publicknowledge.org \\
\whois{Counsel for Amicus Curiae}
}

\title{Brief of Public Knowledge as
\protect\\\emph{Amicus Curiae} in Support of Respondent}

\begin{document}

\maketitle

\romanpagenumbers
\tableofcontents

\part{Table of Authorities}

\tableofauthorities

\clearpage
\arabicpagenumbers

\part{Interest of \emph{Amicus Curiae}}

Public Knowledge is a non-profit organization that is dedicated to preserving
the openness of the Internet and the public's access to knowledge; promoting
creativity through balanced intellectual property rights; and upholding and
protecting the rights of consumers to use innovative technology lawfully. As
part of this mission, Public Knowledge advocates on behalf of the public
interest for a balanced patent system, particularly with respect to new and
emerging technologies.\footnote{Per Supreme Court Rule 37(6), no counsel for a
party authored this brief in whole or in part, and no counsel or party made a
monetary contribution intended to fund the preparation or submission of the
brief. No person or entity, other than \amicus, its members, or its counsel,
made a monetary contribution to the preparation or submission of this brief. Per
Rule 37(3)(a), consent has been granted for the filing of this brief, as
indicated by the blanket consents from counsel for petitioner and counsel for
respondents docketed December 11, 2013.}

Public Knowledge has previously served as \amicus in key patent cases.
\sentence{e.g., i4i; bilski; quanta}.

\fullcite{i4i}
\fullcite{bilski}
\fullcite{quanta}

\part{Summary of Argument}

*

\part{Argument}

\section{The Claims at Issue Are Ineligible Under Section 101 Because They
Effectively Preempt Substantially All Uses of an Abstract Idea}

The issue of this case is whether Petitioner's claims are directed to
patent-eligible subject matter. Although the statute states that ``any new and
useful process, machine, manufacture, or composition of matter'' is eligible for
patenting, \clause{101}, the Court has identified three exceptional fields that
are nevertheless ineligible: laws of nature, physical phenomena, and abstract
ideas. \sentence{e.g., mayo at 1293 (quoting \sentence{diehr at 185})}.

The claims of these patents, as with many patents in the computer technology
field, are full of complex technical language. To facilitate the Court's
analysis of this case, \amicus has prepared a \numlines-line computer program
that implements all the features of one of the claims. Based on this program,
\amicus believes, the claims recite not specialized, technical systems, but
rather a broad, general, simple algorithm that reduces to nothing more than an
abstract idea run on a computer. Because mere application of a general-purpose
computer should not render an otherwise abstract idea patentable, \amicus
urges the Court to find the present claims ineligible for patenting.

\subsection{A \Numlines-Line Computer Implementation of the Patented Technology
Illustrates that the Claims Are Not Meaningfully Limited Beyond an Abstract
Idea}

A great deal of the disagreement in the lower court's fractured decision stemmed
from a disagreement over the nature of the patent claims at issue. One plurality
of the court found the claims to be highly generic, reciting ``a handful of
computer components in generic, functional terms that would encompass any
device'' and unduly preempt an abstract idea. \sentence{see cls-enbanc at 1290
(Lourie, J., concurring)}. Another found those same claims narrowly tailored,
``limited to an implementation that inclueds at least four separate structural
components'' rendering the claim patent-eligible. \sentence{see cls-enbanc at
1307 (Rader, J., concurring-in-part and dissenting-in-part)}.

Indeed, the claims do use complex, technical-sounding language, making them
appear to be directed to a narrowly tailored invention. One of the claims at
issue recites, among other things, a ``communications controller,'' a
``data storage unit,'' and an ``instruction being an irrevocable, time invariant
obligation.'' \sentence{pat375 at claim 26, col. 66-67}.\footnote{Claim 26 of
\inline{pat375} is considered in this brief because it was found patentable by
the greatest number of judges of the lower court decision. \sentence{see
cls-enbanc at 1309 (Rader, Linn, Moore \& O'Malley, JJ., concurring-in-part and
dissenting-in-part); cls-enbanc at 1327 (Newman, J., concurring-in-part and
dissenting-in-part)}. The analysis applied by \amicus to this claim can, of
course, be applied to other claims at issue in this case. For reference, Claim
26 is reprinted in \inline{this at appx:claim}.}

However, upon closer inspection, the Court should find that beneath this veneer
of technical language is a very simple, basic idea being patented. To
demonstrate this, we prepare a computer program that implements all the
features of the claims:

\vskip.5\baselineskip
\inlinebox{\wholeprogram}%
\vskip.5\baselineskip

The program is written in the \textsc{basic} programming language, which was
selected both because it is simple to understand and because it predates the
earliest priority date of the patent. A complete explanation of the working of
this program as it relates to Claim 26 of \inline{pat375} is presented in
\inline{this at appx:code}.

As the Court will observe, the computer program is very short, indicating that
the verbose language of the claims does not in fact demand specific, particular
implementations but rather expansively preempts all uses of a simple, basic
idea.

Certain judges of the Federal Circuit were clearly misled by the language of the
claims and the patent.
One plurality believed that this very claim ``covers the use of a computer and
other hardware specifically programmed to solve a complex problem'' through the
use of ``at least four separate structural components.'' \sentence{cls-enbanc at
1307 (Rader, J.)}. The plurality reviewed the ``at least thirty two figures
which
provide detailed algorithms for the software'' to conclude that ``[l]abeling
this system claim an `abstract concept' wrenches all meaning from those words.''
\sentence{cls-enbanc at 1309 (Rader, J.)}. Another plurality found a
similarly-worded claim ``limited to one that is configured to perform certain
functions in a particular fashion'' and proceeded to review one of the
flowcharts of the patent to suggest that the claims demanded a dizzyingly long
and complex algorithm. \sentence{cls-enbanc at 1318 (Moore, J.,
dissenting-in-part)}. A third plurality concluded that, while they may be based
on an abstract idea, ``the claims here are directed to very specific ways of
doing that,'' proceeding to enumerate seven claim terms from a related claim.
\sentence{cls-enbanc at 1741 (Linn, J., dissenting)}.

The common thread among all of these judges is an assumption, based on technical
language recited in the specification and claims of the patents, that only a
specific, complex, technical computer program could infringe the patents. As the
above \numlines-line computer program demonstrates, this assumption was in
error.

\Amicus is aware that the Court's precedent requires that ``claims must be
considered as a whole.'' \sentence{diehr at 188}. However, this principle does
not warrant reading the claims too narrowly, despite the fact that many of the
judges of the Federal Circuit did just that under the banner of the
claims-as-a-whole proposition.\footnote{%
\sentence{see cls-enbanc at 1298 (Rader, J.);
cls-enbanc at 1315 (Moore, J.) (``My colleagues erroneously apply
\emph{Prometheus}'s `inventive concept' language by stripping away all known
elements from the asserted system claims\ldots.''); cls-enbanc at 1331 (Linn,
J.)
(criticizing Judge Lourie's dissent because ``he actually strips the claims of
their detail and limitations''); see also cls-panel at 1353 note 4 (Linn, J.)
(criticizing the dissenting judge for providing a ``plain English translation''
of the claims)}.} Claims must be read with neither too broad nor too narrow a
scope; instead they must be accorded the scope congruent with their language.

The use of a computer program does
indeed read the claim as a whole. As shown in detail in the appendix, every
claim limitation has been considered and implemented appropriately in the
computer code, so it cannot be said that details or limitations have been
stripped from the claim. \sentence{see this at appx:code; cf. diehr at 188 (``It
is inappropriate to dissect the claims into old and new elements and then to
ignore the presence of the old elements in the analysis.'')}. Furthermore,
because
the computer program is a functional, working implementation of the claim, it
cannot be argued that it is a mere abstraction or generalization of the claims.

Accordingly, Claim 26 of \inline{pat375} is directed not to a complex, technical
system requiring specialized hardware, but rather a simple, basic,
\numlines-line computer algorithm. \Amicus thus turns next to show that this
basic algorithm is an abstract idea not eligible for patenting.

\subsection{Read with Proper Expansiveness, the Claims Cover Substantially All
Computer Implementations of a Basic, Abstract Accounting Idea of Third-Party
Escrow}

The example computer program shows that the asserted claims, though lengthy and
technical in appearance, are actually directed only to a very simple, basic
computer procedure. \Amicus now turns to using this example computer program to
show that the asserted claims are directed only to an abstract idea.

``A principle, in the abstract, is a fundamental truth; an original cause; a
motive; these cannot be patented, as no one can claim in either of them an
exclusive right.'' \sentence{benson at 67 (quoting \sentence{leroy at 175})}.
Claims directed merely to an abstract idea are
not eligible under \inline{101}, because they are the ``basic tools of
scientific and technological work,'' \clause{benson at 67}, which therefore
are ``part of the storehouse of knowledge of all men\ldots free to all men and
reserved exclusively to none,'' \clause{bilski at 3225 (quoting
\sentence{funkbros
at 130}) (omission in original)}. Furthermore, inclusion of ``conventional or
obvious''
post-solution or pre-solution activity to an otherwise abstract idea does not
render a claim eligible, as this Court has held, because otherwise
``a competent draftsman could attach some form of post-solution activity'' to
``transform an unpatentable principle into a patentable process.''
\sentence{flook at 590; see also mayo at 1300 (holding ``conventional steps,
specified at a high level of generality,'' to similarly not confer patent
eligibility)}.

As Respondents have argued, the claim is directed to nothing more than a
general-purpose computer tied to the abstract idea of accounting by a
third-party escrow. The following exposition will consider Claim 26, line by
line, to determine that every claim element is (1) an inherent aspect of this
abstract idea, (2) a standard component of a general-purpose computer, or
(3) insignificant pre- or post-solution activity.


Elements 1--2\footnote{Elements will be referenced by numbers corresponding to
the claim reprinted in the appendix. \sentence{see this at appx:claim}.} of the
claim describe ordinary components of a general-purpose computer. \sentence{see
this at appx:claim:elements 1--2}. The ``communications controller'' and
``first party device'' are broad, general terms that encompass basic computer
components for interacting with users. Furthermore, these two components are
only recited in conjunction with a step of receiving data, which as explained
below is insignificant pre-solution activity.

Elements 3--5 describe basic record-keeping operations inherent in the idea of
third-party escrow. Although the claim language verbosely describes a ``data
storage unit'' with ``information about a first account'' and second account,
the computer implementation demonstrates that these claim elements in fact
require nothing more than storing account balances---that is, recording two
numbers in a computer. \sentence{see this at appx:claim:elements 3--5}.
Certainly one would necessarily store such account information as part of an
escrow service.

Element 6 recites ``a computer,'' and as such only further describes a
general-purpose computer.

Element 7 states that the computer must ``receive a transaction.'' Steps of
obtaining data to be used for processing have been held by this Court and others
to constitute insignificant pre-solution activity. \sentence{see, e.g., mayo at
1297-1298 (treating as pre-solution activity a step of determining a level of
metabolites prior to adjusting a treatment)}. As such, this
claim element does not contribute to the eligibility of the claim.

Element 8 describes two steps to be performed by the computer, both of which are
inherent in the idea of third-party escrow. First, the computer is tasked with
``ensuring that said first party and/or said second party have adequate value''
in their accounts. The computer code shows that this amounts to nothing more
than a comparison, checking whether an account balance is greater than an amount
to be transferred out of that account. \sentence{see this at appx:claim:element
8}. This is a necessary operation performed by a third-party escrow broker, who
must ensure that the parties' accounts can satisfy the desired transaction.

Second, element 8 requires the computer to ``electronically adjust said first
account and said third account.'' This is performed in two lines of computer
code, one of which subtracts from the first account and the other of which adds
to the third account. \sentence{see this at appx:claim:element 8}. Again, this
is inherent in a third-party escrow service, which must adjust account balance
records to account for a transaction.

Element 9 instructs that the computer ``generate an instruction to said first
exchange institution and/or said second exchange institution to adjust said
second account and/or said fourth account.'' Despite the fifty-nine-word length
of this element, it reduces to a single computer operation of printing out a
message describing the transaction that was just completed. \sentence{see this
at appx:claim:element 9}. This elementary output step is post-solution activity
that should not contribute to the eligibility of the claim. \sentence{cf.
flook at 590 (treating as post-solution activity a step of adjusting an alarm
limit in response to a computation)}.

Accordingly, this claim is directed to nothing more than an abstract idea of
third-party escrow, in conjunction with insignificant pre-solution and
post-solution activity, and ordinary, albeit verbosely described, components of
a general purpose computer.

\subsection{The Court Should Disapprove the Preemption of Substantially All
Computer Uses of an Abstract Idea, and Thus Hold the Claims at Issue Ineligible}

Under this Court's precedent, a patent claim is ineligible under \inline{101} if
that claim has the ``practical effect'' of removing an abstract idea from the
public domain.  \sentence{see benson at 71-72}.  As shown in the previous section,
claim 26 of the '375 patent would have the practical effect of removing all uses
of an abstract idea \emph{implemented on a general-purpose computer} from the
public domain. The Court should make clear that abstract ideas, including
algorithms, are reserved to the public, even when implemented on general-purpose
computers.

Holding otherwise would contradict the purpose of the exception and undermine the
justification for patent protection.  Abstract ideas are unpatentable because they
are ``the basic tools of scientific and technological work,'' \clause{benson at
67}, and must remain ``free to all men and reserved exclusively to none,''
\clause{bilski at 3218 (quoting \sentence{funkbros at 130})}. All patents
necessarily remove something from the public domain, but where patents ``grant[]
monopolies over procedures that others would discover by independent, creative
application of general principles,'' they reach impermissibly far. \sentence{see
bilski at 3228}.

The exception's purpose is a practical one:  prohibiting patents on abstract ideas
would be meaningless if ideas could still be patented within certain technical
domains, or with the addition of routine steps. \footnote{Note to ourselves: 
possible analogy to genus and species claims:  once the genus is disclosed, can’t
patent individual species within it, because allowing otherwise would encourage a
rush to cover the genus with individual patents, making the genus patent less
valuable.  (In the analogy the genus patent is the public domain idea.)} Thus, in
\inline{mayo} this Court held that processes reciting laws of nature are not
patentable unless they require ``significantly more'' than the natural law,
thereby providing ``practical assurance'' that the claim is not ``designed to
monopolize the law of nature itself.''  \sentence{mayo at 1297; see also mayo at
1294 (claims must be limited ``in practice'')}.

Implementing an abstract idea like an algorithm on a general-purpose computer does
not provide the practical protection against monopoly that this Court's past cases
require.  Recognizing this, \inline{benson} held that a claim to a method of
programming a general-purpose computer was an abstract idea ineligible for patent
despite the fact that it required computer implementation. \sentence{benson at
71}.  The Court reasoned that because the claimed method had ``no substantial
practical application except in connection with a digital computer,'' the patent
``in \emph{practical effect} would be a patent on the algorithm itself.''
\sentence{benson at 71-72 (emphasis added)}. \iffalse COMMENTED BECAUSE I THINK
IT'S REDUNDANT WITH THE NEXT PARAGRAPH --ANNA Instead, requiring computer
implementation is like adding “conventional activity” to an abstract idea,
\clause{mayo at 1300}, or limiting it to a “particular technological environment,”
\clause{bilski at 3230 (quoting \sentence{diehr at 191-192})}, both of which the
Court rejects as attempts to “circumvent[]” the exception, \clause{mayo at 1294
(quoting \sentence{bilski at 3225})}.  Thus, \inline{flook} held that neither
limiting an abstract idea to use within the petrochemical industry, nor adding
“post-solution activity” like adjusting a threshold value, could render it
patent-eligible. \sentence{parker at 590}. \inline{bilski} held that a method of
hedging was unpatentable even when limited to energy markets. \sentence{bilski at
3231}.  And \inline{mayo} held that adding “well-understood, routine, conventional
activity previously engaged in by researchers in the field” was not a “meaningful
limit.” \sentence{mayo at 1294, 99999}. \fi

An abstract idea ``does not suddenly become patentable subject matter simply by
having the applicant acquiesce to limiting the reach of the patent . . . to a
particular technological use.''  \sentence{diehr at 192 n.14}. Implementing an
abstract idea in the form of an algorithm on a general-purpose computer is a
``well-understood, routine, conventional activity,'' \clause{mayo at 1300}, that
applies the algorithm in a ``particular technological environment,''
\clause{bilski at 3230 (quoting \sentence{diehr at 191-192})}.  The mere fact that
a computer is involved does not make these meaningful limits. Unlike claims to
``an application'' of an abstract idea that ``[do] not seek to pre-empt the use''
of the abstract idea, \clause{see diehr at 187 (finding claim to such an
application patent-eligible)}, requiring implementation on a general-purpose
computer therefore does not limit the claim ``in practice,'' \clause{see mayo at
1294}.\footnote{Algorithms run on a general-purpose computer are also far from the
``industrial process'' the Court found to be a patent-eligible application in
\inline{diehr}. \sentence{see diehr at 187 (describing a process with ``physical
and chemical'' components including filling a mold with raw rubber and opening the
surrounding press at a calculated moment, whose contribution to the art was its
``process of constantly monitoring the actual temperature inside the mold.'')}.}

The physical nature of a computer does not change this.  Holding otherwise would
``exalt[] form over substance,'' \clause{see flook at 590}, and endorse an
untenable legal fiction.  The Federal Circuit case \inline{alappat}, which Alice
and some of its \emph{amici} cite approvingly, \iffalse COMMENTED AS UNNECESSARY
EXPLANATION --ANNA illustrates why this Court has previously rejected such “rigid
line-drawing.” \sentence{alappat; bilski at 99999}.  \inline{alappat} drew a
formalistic distinction between general-purpose and special-purpose computers
wherein ideas implemented on the second were patentable, then relied on the
fiction that a \fi found that a general-purpose computer ``in effect'' becomes a
patent-eligible special-purpose machine once it is programmed.  \sentence{see
alappat at 1545}. But this is like saying that a television screen is
``special-purpose'' at the moment it displays one frame from a movie, or, as Chief
Judge Archer pointed out at the time, that a composer may patent the ``structure''
of a song as recorded on a compact disc.  \sentence{see alappat at 1553-1554
(Archer, C.J., dissenting)}. \inline{alappat}'s distinction between ``apparatus''
and unpatentable ``mathematics,'' \clause{alappat at 1545}, is a fiction with no
relevance to general-purpose machines that function through repeated logical
operations. \iffalse COMMENTED BECAUSE I COULDN'T FIND THIS AND IT'S NOT NECESSARY
--ANNA and moreover this Court in \inline{bilski} rejected the idea that use of
computer technology is categorically sufficient to confer patent protection.
\sentence{see bilski at 3227}. \fi

Like pencil and paper, programmable computers are a general-purpose innovation
technology.  Just as most of us cannot solve, say, long division problems of a
certain difficulty without writing them down, computers are capable of tasks we
humans cannot practically perform unaided.  This is why general-purpose computers
are so useful:  they can be put to any task for which they can be programmed.
Advances in computer hardware are of course themselves eligible for patent
protection.  \sentence{see cls-enbanc at 1292 (Lourie, J., plurality op.)
(observing in dicta that computers \emph{per se} are ``surely patent-eligible
machines'')}.  But so are machines made of graphite and wood.  This does not mean
that the abstract idea of long division is eligible for protection because it can,
or even sometimes practically must, be written down.

The basic tools of innovation remain basic tools, even when they are implemented
on general-purpose technologies, or must be implemented on general-purpose
technologies.  To hold otherwise would allow processes to be patentable when run
on a computer that we could not remove from the public domain when done by hand.
This would hamstring innovators by limiting them to a previous generation's
general-purpose technologies—pencils, calculators—and reserving the use of
abstract ideas that can be most effectively, or even only, performed on
general-purpose computers to those who patent them first.  Computers are in
widespread use today, and are effectively unavoidable. Condemning the public to
resort to pencil and paper to avoid patent infringement is untenable, and goes
against the purpose of the abstract idea exception.

This case is the Court's opportunity to reject the fiction that computers'
physicality effectively limits the ideas they implement, continue its reasoning in
\inline{mayo} and \inline{bilski}, and clearly hold that abstract ideas cannot be
patented even when they are implemented on general-purpose computers.

\section{The Court Should Proactively Clarify the Law of Subject Matter
Eligibility in Order to Avoid Further Errors Relating to Abstract Ideas}

The Supreme Court has taken numerous subject matter eligibility cases recently.
It does so because the Federal Circuit is in a confused state about the law of
\inline{101}, primarily because a small faction of that court repeatedly applies
incorrect analytical techniques to improperly find patents eligible even when
this Court's precedents demand otherwise.

To clearly enunciate the law for the Federal Circuit and to prevent the need for
further appeals, this Court should explicitly reject those improper analytical
techniques, some of which have been catalogued below.

\subsection{The Court Should Enunciate the Inappropriateness of Using
Specification Details to Evaluate Subject Matter Eligibility}

In assessing whether a claim is ineligible under \inline{101}, courts must
consider the entire breadth of the claim. Claims directed to an abstract idea
will still cover specific, concrete implementations of that abstract idea, so
the mere fact that a claim covers a concrete implementation is no indicator that
a claim is directed to eligible subject matter.

Nevertheless, certain judges of the Federal Circuit persistently err by relying
on specific examples to find patent claims eligible. In the present case, the
plurality opinion justified its finding that the system claims of the patents at
issue were eligible, by selecting a complex-looking flowchart from the
specification to point to the supposed complexity and concreteness of the claim.
By doing so, they failed to contemplate the possibility that other, simpler,
abstract ideas were \emph{also} covered by that same claim---ideas such as the
14-line computer program presented in this brief.

Ironically, those same judges of the Federal Circuit criticize their opposed
colleagues for failing to read the ``claims as a whole.'' It is in fact those
opposed colleagues who have actually read the claims as a whole, contemplating
the vast scope of what they cover. It is that plurality of the Federal Circuit,
instead, who fails to read the claims as a whole, focusing wrongly on specific
examples and obfuscatory language that misleadingly make abstract ideas appear
patentable.

\subsection{The Court Should Reaffirm its Longstanding View that Mere Drafting
Decisions, such as Choosing Between System and Method Claims, Do Not Affect
Subject Matter Eligibility}

The formalistic approach favored by some judges of the Federal Circuit lends to
easy circumvention by clever patent drafting. For example, the suggestion that
the method claims in the present case are ineligibile, while system claims
directed to the same technology are eligible, simply encourages patent
applicants to use system claims in order to skirt the abstract ideas test.

Granting such weight to mere formal drafting practices ignores the basic
rationale behind the Supreme Court's exceptions to \inline{101}. In explaining
the basis for the three exceptions to \inline{101}, this Court has applied the
fundamental principle that patents must ultimately incentivize innovation. While
patents on many inventions do serve this principle, patents to abstract ideas,
laws of nature and physical phenomena would in fact deter innovation by taking
away those ``basic tools of research available to all.''

Several judges of the Federal Circuit ignore this basic rationale. Judge Rader,
for example, has intimated that the three exceptions to \inline{101} are
essentially tautological, because one ``cannot invent an abstract idea, law of
nature or physical phenomenon'' since they have been around the whole time.

This unduly narrow, formalistic view of the exceptions to \S~101 fails to
adequately protect the concerns about incentives for innovation explicitly
relied upon by this Court. Under Judge Rader's view, mere addition of even the
most insignificant step to an otherwise abstract method would suddenly make that
abstract method patentable, because the combination would not have existed
before. The Court has specifically denounced this possibility, in holding
numerous times that insignificant post-solution activity and pre-solution
activity cannot render an otherwise abstract idea patentable.

\subsection{Recitation of Basic, Widely Available Platform Technologies,
Regardless of Detail, Cannot Render an Abstract Idea Patentable}

The Federal Circuit repeatedly cites recitations of basic general purpose
computing hardware as evidence that a claim is directed to eligible
subject matter under \inline{101}. This is often done by overstating this
court's dicta in \inline{bilski}, that the ``machine or transformation'' test is
an ``important clue'' in assessing subject matter eligibility.

The Court should clarify that mere recitation of general purpose platform
technologies, such as general purpose computers, cannot render an otherwise
ineligible claim eligible. Such a holding would be consistent with this Court's
precedent, and more importantly would strongly advance the principles of
incentivizing innovation, by protecting those ``basic tools of innovation''
meant to be ``available to all.''

As an analogy, consider a claim directed to the basic idea of addition,
performed with paper and pencil. The paper and pencil could be described in
great detail:
\begin{quote}
Drawing one or more numerical figures, with a pencil comprising a wooden shaft
substantially in the shape of a hexagonal prism, the wooden shaft surrounding a
cylindrical graphite barrel, the wooden shaft having a distal end including a
rubber eraser, the wooden shaft further having a proximal end sharpened to
thereby expose a portion of the cylindrical graphite barrel.
\end{quote}
Such a claim would certainly satisfy the machine-or-transformation test (a
pencil is a machine of sorts, and the adherence of graphite to paper would
constitute transformation of matter, among other things), but certainly such a
claim would not be eligible subject matter, regardless of the level of detail.
This is because paper and pencil are the basic tools of invention. To permit the
patenting of abstract ideas merely tied to such basic tools would be tantamount
to permitting the patenting of those abstract ideas alone.

Certain judges of the Federal Circuit criticize this approach, believing that it
improperly imports questions of novelty and obviousness into \inline{101}.
However, as this Court's precedent makes clear, this is not the case.
\sentence{see flook}.

\part{Conclusion}

For the foregoing reasons, \amicus respectfully submits that the Court
should affirm the district court.

\signature

\appendix{Implementation of Claim 26 of the '375 Patent in \Numlines Lines of
Computer Code}
\bblabel[Appendix \#]{appx:code}

The following \numlines-line computer program, written in the \textsc{basic}
programming language, implements Claim 26 of the '375 Patent.

\inlinebox{\wholeprogram}

The subsequent text reviews the elements of the claim in detail and explains how
a general-purpose computer, running the above computer program, would satisfy
all the elements of the claim. For convenience, the entirety of the claim is
reprinted in the next appendix.

\claimbox{preamble}{\claimelt0}

The preamble recites that the claim covers a general purpose computing system,
called a ``data processing system'' by the claim language. The recitation that
the system is ``to enable the exchange of an obligation'' is a statement of
field of use or intended use, which should not contribute to the scope of the
claim.
\sentence{see bilski at 3231 (``[L]imiting an abstract idea to one field of
use\ldots did not make the concept patentable.''); mpep at S 2103/IC
(instructing that ``statements of intended use or field of use'' ``may raise a
question as to the limiting effect of the language in a claim'')}.

\claimbox{elements 1--2}{\claimelt1\par\claimelt2}

These elements recite general hardware inherent in a general purpose computer. A
``communications controller'' broadly refers to a component of a computer that
receives and processes communications, and a ``first party device'' could refer
to any computer hardware. A standard keyboard could potentially satisfy this
limitation.

\codeandclaimbox{elements 3--5}{\claimelt3\par\claimelt4\par\claimelt5}{1}{2}

These elements of the claim simply require that a computer store two numbers
representing account balances. The ``data storage unit'' might be any computer
storage component, such as a hard disk or memory. The ``information about'' the
first and third accounts broadly encompass any account information, such as an
account balance.

The recitations that the information be stored ``independent from'' various
accounts maintained by exchange institutions are simply statements of intended
use, which should not contribute to the patentability of the claim. Petitioners
have never suggested that the external exchange institutions are necessary
parties to infringement of their claims. Furthermore, so long as the two stored
numbers reflect actual account balances in external banks, the ``independent
from'' limitations are satisfied.

The computer code implements these elements of the claim by instructing a
computer to store two account balances, into variables named \emph{account1} and
\emph{account3}.

\claimbox{element 6}{\claimelt6}

This element is simply further recitation of details about a general purpose
computer. Any computer would necessarily be coupled to a data storage unit, so
that it might access data for processing, and further be coupled to a
communications controller, so that it may receive and output information.

\codeandclaimbox{element 7}{\claimelt7}{3}{3}

According to this element, the computer receives a ``transaction.'' An exchange
of money between two accounts is one type of transaction. Thus, this element
requires nothing more than receipt of an instruction to transfer money between
two accounts.

The computer code implements this by requesting input of an amount of money to
transfer between the first and third account. Upon running this line of code,
a computer would print out a prompt message, and then await an outside user to
enter a number indicating the amount of money to transfer. The amount to
exchange is stored in a variable named \emph{exchange}.

\codeandclaimbox{element 8}{\claimelt8}{4}{6}

This element describes two operations. First, a computer must check that at
least one of the accounts has a large enough balance to permit the desired
transfer of money (``ensuring that said first party\ldots ha[s] adequate value
in
said first account''). Second, the computer must record the transfer by
adjusting the balances of the accounts (``electronically adjust said first
account and said third account'').

Note the substantial presence of inoperative language in this claim element. The
recitation ``in order to effect an exchange obligation arising from said
transaction between said first party and said second party'' does nothing more
than reiterate that the computer is transferring money between accounts.
Furthermore, the claim recites that the computer must ensure ``adequate value in
said first account and/or said third account,'' and the disjunctive ``and/or''
means that the claim element is satisfied if only one of those accounts is
checked. \sentence{see mpep at S 2103/IC (``Language that suggests or makes
optional but does not require steps to be performed\ldots does not limit the
scope of a claim or claim limitation.'')}.

The computer code implements the step of checking the account balances at line
40, which halts execution (with \textsc{stop}) if the balance of
\emph{account1} is less than the amount to be exchanged. The code implements the
step of effecting the transfer at lines 50--60, which deducts the amount to be
exchanged from \emph{account1} and adds that amount to \emph{account3}.

\codeandclaimbox{element 9}{\claimelt9}{7}{7}

This claim element requires only that a computer display an instruction to
perform the desired transfer of money. The claim element recites ``an
instruction to said first exchange institution and/or said second exchange
institution,'' but the disjunctive ``and/or'' means that a single instruction
suffices. Similarly, the recitation of an instruction ``to adjust said second
account and/or said fourth account'' only requires an instruction with regard to
a single account.

The requirement that the instruction be ``an irrevocable, time invariant
obligation'' is merely a statement of intended use that should not contribute to
the patentability of the claim. An instruction is simply a text, and the
recipient of the instruction chooses whether to treat that text as irrevocable
or time-invariant. Although this claim language could plausibly have been
defined in the specification to require some sort of special format for the
instruction, Petitioners have never identified any such special definition in
any of their briefs to this Court, the Federal Circuit, or the district
court,\footnote{The district court briefs reviewed are identified on the docket
as Documents Nos.\ 53, 54, 68, 95, and 99. The Federal Circuit briefs reviewed
are identified on the docket as Documents Nos.\ 22, 33, 41, and 194.}
and the text of the
specification contains neither term outside of the claims. Furthermore, even if
these terms did have some special meaning, it would only dictate the content of
the instruction text, and content of text does not contribute to patentability.

The computer code implements this element by causing a computer to print an
instruction to adjust the second account. The instruction directs the
first institution to deduct the amount \emph{exchange} from the account.

\appendix{Claim 26 of the '375 Patent}
\bblabel[Appendix \#]{appx:claim}
\addtotoa{pat375}

\emph{Numbers, in square brackets, have been inserted before each element of the
claim, to assist in referring to claim elements within the brief.}

\wholeclaim

\end{document}

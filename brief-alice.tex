\documentclass{scotus}
\usepackage{bluebook}

\defcase{i4i}{
p=Microsoft Corp.,
d=i4i Limited Partnership,
cite=131/S. Ct./2238,
year=2011
}

\defcase{bilski}{
p=Bilski,
d=Kappos,
cite=130/S. Ct./3218,
year=2010
}

\defstatcode{112p2}{
cite={35/U.S.C./S 112, \P~2},
year=2011
}

\defcase{quanta}{
p={Quanta Computer, Inc.},
d=LG Electronics Corp.,
cite=553/U.S./617,
year=2008
}



%
% If you do not have these fonts, comment the next few lines out by changing
% \iftrue to \iffalse
%
\iftrue
    \usepackage{fontspec}
    \setmainfont[BoldFont=ITC Century Std-Bold,Ligatures=TeX]{%
        Century Expanded LT Std%
    }
    \def\titlefont{\fontspec{Cloister Black}\LARGE}
\fi

\docket{13-298}
\casecaption{
\textsc{Alice Corporation Pty. Ltd.},\\\The{Petitioner,}\\v.\\
\textsc{CLS Bank International and CLS Services Ltd.},\\
\The{Respondent.}
}

\posture{On Writ of Certiorari \\
to the United States Court of Appeals \\
for the Federal Circuit}

\counsel{
\textsc{Charles Duan} \\
\counselofrecord \\
\textsc{Public Knowledge} \\
1818 N St NW, Suite 410 \\
Washington, DC 20036 \\
(202) 861-0020 \\
cduan@publicknowledge.org \\
\whois{Counsel for Amicus Curiae}
}

\title{Brief of Public Knowledge as
\protect\\\emph{Amicus Curiae} in Support of Respondent}

\begin{document}

\maketitle

\romanpagenumbers
\tableofcontents

\part{Table of Authorities}

\tableofauthorities

\clearpage
\arabicpagenumbers

\part{Interest of \emph{Amicus Curiae}}

Public Knowledge is a non-profit organization that is dedicated to preserving
the openness of the Internet and the public's access to knowledge; promoting
creativity through balanced intellectual property rights; and upholding and
protecting the rights of consumers to use innovative technology lawfully. As
part of this mission, Public Knowledge advocates on behalf of the public
interest for a balanced patent system, particularly with respect to new and
emerging technologies.

Public Knowledge has previously served as \emph{amicus} in key patent cases.
\sentence{e.g., i4i; bilski; quanta}.

\fullcite{i4i}
\fullcite{bilski}
\fullcite{quanta}

\part{Summary of Argument}

*

\part{Argument}

\section{A Plurality of the Federal Circuit Repeatedly Misapplies the Court's
Precedent to Hold Abstract Ideas Patentable}

\subsection{The System Claims of the Patents at Issue, Held Eligible Under
\S~101, Are in Fact Directed to an Abstract Idea Implementable in 14 Lines of
Computer Code}

[Like WildTangent brief]

\subsection{Because of the Cloak of Dense, Technical Claim Language, the Federal
Circuit Misapprehended the Abstractness and Generality of the Claim}

[Like WildTangent brief]

\subsection{Permitting the Patentability of Such Claims Would Encourage Abuse of
the Patent System, Impair Public Notice, and Reduce Innovation}

[Like WildTangent brief]

\section{To Rectify These Ongoing Errors, The Court Should Issue Certain
Holdings to Explicitly Disapprove Analytical Techniques Used by the Federal
Circuit}

The Supreme Court has taken numerous subject matter eligibility cases recently.
It does so because the Federal Circuit is in a confused state about the law of
\inline{101}, primarily because a small faction of that court repeatedly applies
incorrect analytical techniques to improperly find patents eligible even when
this Court's precedents demand otherwise.

To clearly enunciate the law for the Federal Circuit and to prevent the need for
further appeals, this Court should explicitly reject those improper analytical
techniques, some of which have been catalogued below.

\subsection{Lower Courts Must Evaluate Subject Matter Eligibility Based on the
Entire Scope of the Claim, Without Improper Reliance on the Specification}

In assessing whether a claim is ineligible under \inline{101}, courts must
consider the entire breadth of the claim. Claims directed to an abstract idea
will still cover specific, concrete implementations of that abstract idea, so
the mere fact that a claim covers a concrete implementation is no indicator that
a claim is directed to eligible subject matter.

Nevertheless, certain judges of the Federal Circuit persistently err by relying
on specific examples to find patent claims eligible. In the present case, the
plurality opinion justified its finding that the system claims of the patents at
issue were eligible, by selecting a complex-looking flowchart from the
specification to point to the supposed complexity and concreteness of the claim.
By doing so, they failed to contemplate the possibility that other, simpler,
abstract ideas were \emph{also} covered by that same claim---ideas such as the
14-line computer program presented in this brief.

Ironically, those same judges of the Federal Circuit criticize their opposed
colleagues for failing to read the ``claims as a whole.'' It is in fact those
opposed colleagues who have actually read the claims as a whole, contemplating
the vast scope of what they cover. It is that plurality of the Federal Circuit,
instead, who fails to read the claims as a whole, focusing wrongly on specific
examples and obfuscatory language that misleadingly make abstract ideas appear
patentable.

\subsection{The Exceptions to \S~101 Are No Mere Tautologies, but
Rather Arise from the Utilitarian Principle of Patents}

In explaining the basis for the three exceptions to \inline{101},
this Court has applied the fundamental principle that patents must ultimately
incentivize innovation. While patents on many inventions do serve this
principle, patents to abstract ideas, laws of nature and physical phenomena
would in fact deter innovation by taking away those ``basic tools of research
available to all.''

Several judges of the Federal Circuit ignore this basic rationale. Judge Rader,
for example, has intimated that the three exceptions to \inline{101} are
essentially tautological, because one ``cannot invent an abstract idea, law of
nature or physical phenomenon'' since they have been around the whole time.

This unduly narrow, formalistic view of the exceptions to \S~101 fails to
adequately protect the concerns about incentives for innovation explicitly
relied upon by this Court. Under Judge Rader's view, mere addition of even the
most insignificant step to an otherwise abstract method would suddenly make that
abstract method patentable, because the combination would not have existed
before. The Court has specifically denounced this possibility, in holding
numerous times that insignificant post-solution activity and pre-solution
activity cannot render an otherwise abstract idea patentable.

\subsection{Recitation of Basic, Widely Available Platform Technologies,
Regardless of Detail, Cannot Render an Abstract Idea Patentable}

The Federal Circuit repeatedly cites recitations of basic general purpose
computing hardware as evidence that a claim is directed to eligible
subject matter under \inline{101}. This is often done by overstating this
court's dicta in \inline{bilski}, that the ``machine or transformation'' test is
an ``important clue'' in assessing subject matter eligibility.

The Court should clarify that mere recitation of general purpose platform
technologies, such as general purpose computers, cannot render an otherwise
ineligible claim eligible. Such a holding would be consistent with this Court's
precedent, and more importantly would strongly advance the principles of
incentivizing innovation, by protecting those ``basic tools of innovation''
meant to be ``available to all.''

As an analogy, consider a claim directed to the basic idea of addition,
performed with paper and pencil. The paper and pencil could be described in
great detail:
\begin{quote}
Drawing one or more numerical figures, with a pencil comprising a wooden shaft
substantially in the shape of a hexagonal prism, the wooden shaft surrounding a
cylindrical graphite barrel, the wooden shaft having a distal end including a
rubber eraser, the wooden shaft further having a proximal end sharpened to
thereby expose a portion of the cylindrical graphite barrel.
\end{quote}
Such a claim would certainly satisfy the machine-or-transformation test (a
pencil is a machine of sorts, and the adherence of graphite to paper would
constitute transformation of matter, among other things), but certainly such a
claim would not be eligible subject matter, regardless of the level of detail.
This is because paper and pencil are the basic tools of invention. To permit the
patenting of abstract ideas merely tied to such basic tools would be tantamount
to permitting the patenting of those abstract ideas alone.

Certain judges of the Federal Circuit criticize this approach, believing that it
improperly imports questions of novelty and obviousness into \inline{101}.
However, as this Court's precedent makes clear, this is not the case.
\sentence{see flook}.

\part{Conclusion}

For the foregoing reasons, \emph{amicus} respectfully submits that the Court
should affirm the district court.

\signature

\end{document}
